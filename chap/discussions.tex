\section{Discussion}
\label{sec:discussion}

\subsection{Scalar Multiplication In Other Cryptographic Libraries}
Besides the OpenSSL library, there are many cryptographic libraries that implement the scalar multiplication.
But not all of them use the wNAF representation.
We introduce some implementations of the scalar multiplication in some popular cryptographic libraries.

The mbed TLS \cite{polarssl} is an open source SSL library licensed by ARM Limited.
It does not use the wNAF algorithm to represent the scalar.
It uses a comb method that generates a sequence of bit-strings to represent the scalar $k$ so that every bit-string represents an odd number based on modifying the method in \cite{Hedabou2004ACM}.
When computing the scalar multiplication, it traverses the sequence and directly indexes the value in the precomputed points to perform the addition.
It does not use the invert function, 
and this representation is hard to attack by the cache side channels because all the elements of the sequence are not zero. 

Libgcrypt is a cryptography library developed as a separated module of GnuPG.[3] It can also be used independently of GnuPG, but depends on its error-reporting library Libgpg-error[4].




