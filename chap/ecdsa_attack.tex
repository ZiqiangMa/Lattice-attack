\section{Attacking ECDSA}
\label{sec:attack}
In this section, we propose a new method to retrieve the private key of ECDSA.
First, we analyse the invert function in the scalar multiplication with wNAF representation in the ECDSA algorithm.
 Then we use the invert function for improving the cache side channel attacks, from which the sign of the non-zero digits of $k$ is determined.
After that, we make use of the obtained information to recover the consecutive bits at the position of every non-zero digits of $k$.
Finally, we construct a HNP instance by the consecutive bits and transform it to the problem of solving the SVP/CVP in some lattice with the lattice reduction algorithms.

\subsection{Attacking wNAF Through the Cache Side Channels}
\label{data_get}
First recalling the implementation of scalar multiplication in OpenSSL, it uses the invert function to
  compute the inverse element of a number.
Applying this function, the space of precomputed points is saved by half, only need to store $\{G, 3G, ..., (2^{w} - 1)G\}$.
As showed in Algorithm~\ref{alg:smopenssl},
it iterates $k$ from the most significant digit to the least significant digit.
In each digit it performs a double function.
While if the digit is not zero, it first determines whether the invert function is executed.
 If the sign of the digit is opposite to the prior one, the invert function is called.
Then it indexes from the precomputed points using the absolute value of the digit and performs an addition operation.

Originally, the vulnerability comes from the double and add function.
The double function is called at every digit, and this reveals the digit sequence.
The add function is called just when the digit is not zero,
  which makes it possible to distinguish whether the digit is zero or not.
  That reveals the position of all the non-zero digits.
However,
 the invert function is also vulnerable, because it is called conditionally, either.
Only in the condition that the sign of the non-zero digit is opposite to the previous, the invert function will be performed.
Because we know the first non-zero digit is positive,
   the sign of all the non-zero digit can be deduced based on the execution of the invert function.


We use a spy process to monitor one memory line of the code of the double, add and invert functions while computing the scalar multiplication.
The time is divided into slots, and in each slot
  the spy determines whether the three functions are performed or not by monitoring the cache hits/misses.
 Then we can obtain a "double-add-invert" chain.
According to the "double" and "add" in this chain, we can determine whether each digit of $k$ is zero or not.
  That is what the original Flush + Reload attack does.
Also, based on the "invert" in this chain, we can infer the sign of each non-zero digit.


Specific to the OpenSSL implementation (especially the OpenSSL1.1.1b used in this paper),
  the double, add and invert function are implemented using  \verb+EC_POINT_dbl()+, \verb+EC_POINT_add()+ and \verb+EC_POINT_invert()+.
 The original cache side channel attack monitors the \verb+EC_POINT_dbl()+ and \verb+EC_POINT_add()+ function.
So as in our attack, we improve the original cache side channel attack by adding a new monitor to the \verb+EC_POINT_invert()+ function.
We use a spy process to monitor one memory line of the code of the three functions.
 In each time slot
  the spy determines whether the three functions are performed or not.
Then we get a new "double-add-invert" chain through the cache side channel attack.

%首先回顾标量乘法的实现,原始实现。
%然后invert函数在里面是如何使用的,起到了什么作用。
%漏洞在哪里,怎么样去利用。
%OpenSSL中他们的实现,
%我们在攻击中怎么做的
%然后得到了什么


When we use the "double-add-invert" chain to extract the digits of $k$,
the "double" represents the double function is called,
and the "add" represents the double and add functions are called both.
The "invert" represents the invert function is called.
Therefore,
 the "double" appears meaning that $k_i$ is zero
and the "add" appears meaning that $k_i$ is not zero.
 Then we use the "invert" to determine the sign of the $k_i$.
The sign of the $k_i$ is related to the previous non-zero digit.
First, the "invert" comes out together with "add".
 Then, if the "invert" appears, it represents that the sign of this digit $k_i$ is opposite to the previous non-zero digit.
 While if the "invert" does not appear when "add" comes out, it represents that
the sign of this digit $k_i$ is the same as the previous non-zero digit.
 For the wNAF representation of $k$, the most significant digit is always positive.
 Thus we can determine the sign of all non-zero digits.
In this way, we obtain all the positions of the non-zero digits and the sign information of them.


% In our work, we implement the Flush+Flush attack to obtain the "double-add-invert" chain because it has a better precision.
%%这里还需要再多说两句,为什么要精度更高,因为reload时间长,多监控一个函数,在一个slot中花费时间更多, 容易造成失误。 flush的时间短,slot可以更短。


%openssl的实现是XXXX,
%传统的攻击对double 和add进行监控
%我们增加对invert进行监控
%然后如何得到digit的正负信息
%在我们的实现中,通过对比f_f和f+r 选择f+f进行攻击


%问题是:标量乘法运算的起始位置很重要。

%符号:sign
%这一节用不用一个图来说明add-double链


\subsection{Consecutive Bits Recovery}
\label{data_proc1}
In this section, we introduce how to recover the consecutive bits at the position of every non-zero bits for the ephemeral key $k$, exploiting the information obtained from the side channel.

First, we denote the wNAF representation of $k$ as $k = \sum{k_{i}2^{i}}$,
 and the binary representation as $k = \sum{b_{i}2^{i}}$.
From the Algorithm \ref{alg:wnaf} we know that the position of $k_i$ in the wNAF representation is the same as $b_i$ in the binary representation.
 When we know the information about whether $k_i$ is zero and the sign of the non-zero $k_i$, we can simply determine some bits of $k$.
 For example if we obtain the sign of the first non-zero $k_j$, we can infer that $b_j$ is one and $b_i$ is zero for $0\leq i<j$.
 But for arbitrary non-zero digits, it can not determine whether the bit is zero or one.

 Let $m$ and $m + l$ be the position of two consecutive non-zero digits of the wNAF representation, and $w$ be the window size.
 That is $k_m, k_{m+l} \neq 0$ and $k_{m+i} = 0$ for all $0 < i < l$.
 We analyse the transformation method between the binary and wNAF representation, getting the following result:
 \begin{align}
 &b_{m+l} = \left\{
 \begin{aligned}
 	&0,\,\;\ \   k_m < 0 \\
 	&1,\,\;\ \   k_m > 0
 \end{aligned}
 \right.   \\
 &b_{m+i} = \left\{
 \begin{aligned}
 	&0,\,\;\ \   k_m > 0 \\
 	&1,\,\;\ \   k_m < 0
 \end{aligned}
 \right.
 ,\ \ \ \ \  w \leq i \leq l-1
 \end{align}
%这里的公式有漏洞,需要加上最低位即m最小时的bm

In this way, at the position of every non-zero digit we can obtain $l - w + 1$ consecutive bits of $k$.
For the wNAF representation, every non-zero digit is followed by at least $w$ zero values.
 The average number of non-zero digits of $k$ is $\lfloor\log_{2}{q}\rfloor /(w+2)$.
 The average distance between consecutive non-zero digits is $w+2$, i.e. on average $l = w + 2$.
 This means we can obtain $3$ consecutive bits on average at the every non-zero digit.
Thus, on average we can obtain $3\lfloor\log_{2}{q}\rfloor /(w+2)$ bits of the ephemeral key $k$ in total.
For the secp256k1 curve implemented in OpenSSL, $\lfloor\log_{2}{q}\rfloor = 256$, $w = 3$.
 Also, as introduced previous, the scalar multiplication uses $k+q$ instead of $k$ in most cases,
  so the total number of bits per signature we obtain is $3(\lfloor\log_{2}{q}\rfloor +1)/(w+2) = 154.2$.
In theory, two signatures would be enough to recover the 256-bit private key as $2\times 154.2 = 308.4 > 256$.

%这里再加一小段,理论上[文献x]说,只要知道连续的2bit信息,就能够用来构造格攻击恢复密钥。
%因此,对侧信道得到的数据使用我们的方法恢复出来的连续比特位,全部都可以被用到后续的攻击中

%首先,k的二进制表示和k的wnaf表示
%他们的位是对应的。
%当我们知道了digit的正负消息后,简单的,我们就可以判断出某些二进制位是多少了
%然后我们来看,怎么去确定
%得到一个公式:
%
%然后我们就得到了连续的bit信息
%通过一个签名,我们能够获得的bit数为:


\subsection{Constructing the Lattice Attack}
\label{data_proc2}
In this section we transform the problem of recovering the private key to the HNP instance, and further convert to the CVP/SVP instance in a lattice.
Our method is based on the analysis from \cite{Nguyen2002}.
But we make some improvements on it.
First the length of the consecutive bits used to construct the lattice is variable while the prior work fixes the length, which may lose some information.
Second  in our method the position of consecutive bits is arbitrary in the ephemeral key and does not need to be fixed, while the prior work needs all the consecutive bits at the same position.
Finally from one signature we obtain multiple sequence of consecutive bits and all of them can be used for constructing the lattice as long as the length of the sequence is satisfied, while the prior work only generates one sequence of consecutive bits for one signature.

To construct a HNP instance using arbitrary consecutive bits, we need to use the following theorem\cite{Nguyen2002}:
\begin{theorem}
 \label{theorem1}
There exists a polynomial-time algorithm which, given $A$ and $B$ in $[1, q]$, finds $\lambda \in Z^{*}_{q}$ such that
$$
|\lambda |_q < B  \ \  \text{and} \ \  |\lambda A|_q \leq q/B .
$$
\end{theorem}
The value of $\lambda$ can be computed exploiting the continued fractions.

Recall the ECDSA signature, $s = k^{-1} (h(m) + r\cdot\alpha) \bmod q$.
We rewrite it as
\begin{equation}
\label{sig}
\alpha rs^{-1} = k - s^{-1}h(m)  \ \  \text{mod} \ \ q.
\end{equation}
Then assume that we have the $l$ consecutive bits of $k$ with the value of $a$, starting at some known position $j$.
 So $k$ is represented as $k = 2^{j}a + 2^{l+j}b +c$ for $0 \leq a \leq 2^l -1$, $0\leq b \leq q/2^{l+j}$ and $0 \leq c < 2^j$.
 We apply the theorem with $A = 2^{j+l}$ and $B = q2^{-j-l/2}$, to obtain $\lambda$ such that
$$
|\lambda |_q < q2^{-j-l/2}  \ \  \text{and} \ \  |\lambda 2^{j+l}|_q \leq q/2^{j+l/2} .
$$
Plugging the value of $k$ and multiplying by $\lambda$, Equation \ref{sig} is transformed to
$$
\alpha r\lambda s^{-1} = (2^{j}a - s^{-1}h(m))\lambda +(c\lambda + 2^{l+j}b\lambda)  \ \  \bmod \ \ q.
$$

 Let
 \begin{equation}
 \label{tu}
 \left\{
 \begin{aligned}
 	&t = \lfloor r\lambda s^{-1} \rfloor_q    \\
 	&u = \lfloor (2^{j}a - s^{-1}h(m))\lambda \rfloor_q
 \end{aligned}
 \right.
 \end{equation}
We then have that
 \begin{equation}
\label{lattice}
    |\alpha t - u|_q < q/2^{(l/2-1)}
\end{equation}
This way, we transform to the a HNP instance.

In practice, OpenSSL uses $k+q$ as the ephemeral key. So the Equation \ref{tu} remains the same, but the Inequation \ref{lattice} turns into
 \begin{equation}
\label{lattice2}
    |\alpha t - u|_q < q/2^{(l/2-\log_{2}{3})}
\end{equation}

Note that, the Equation \ref{lattice2} represents that the $l/2-\log_{2}{3} -1$ most significant bits of $\lfloor\alpha t_i\rfloor$ is $u_i$, based on the definition of the HNP.
So it should satisfy that $l/2-\log_{2}{3} -1 \geq 1$, i.e. $l > 7$.
That means the length of the consecutive bits used to  construct the HNP instance should be larger than $7$,
although we could use all the partitions of the consecutive bits of the ephemeral key in theory.

Next we turn the HNP instance into the lattice problem.
We use $d$ triples $(t_i, u_i, l_i)$ to construct a $d+1$ dimensional lattice $L(B)$ spanned by the rows of the following matrix:
$$B =
\left(
  \begin{array}{ccccc}
    2^{l_1+1}q & 0 & \cdots & 0 & 0 \\
    0 & 2^{l_2+1}q & \ddots & \vdots & \vdots \\
    \vdots & \ddots & \ddots & 0 & \vdots \\
    0 & \cdots & 0 & 2^{l_d+1}q & 0 \\
    2^{l_1+1}t_1 & \cdots & \cdots & 2^{l_d+1}t_d & 1 \\
  \end{array}
\right).
$$
Considering the vector $\textbf{x} = (2^{l_1+1}\alpha t_1 \bmod q, ..., 2^{l_d+1}\alpha t_d \bmod q, \alpha)$,
 it can be obtained by multiplying the last row vector of $B$ by $\alpha$ and then subtracting appropriate multiples of the first $d$ row vectors.
Thus the vector $\textbf{x}$ belongs to $L(B)$.
Let the vector $\textbf{u} = (2^{l_1+1}u_1, ..., 2^{l_d+1}u_d, 0)$,
 so the distance between $\textbf{x}$ and $\textbf{u}$ is $\|\textbf{x} - \textbf{u}\| \leq q\sqrt{d+1}$.
While the lattice determinant of $L(B)$ is $2^{d + \sum{l_i}}q^d$,
 thus the vector $\textbf{x}$ is very close to the vector $\textbf{u}$.
By solving the CVP problem with the input $B$ and $\textbf{u}$, the vector $\textbf{x}$ is revealed,
 hence the private key $\alpha$ is retrieved as it is the last element of the vector $\textbf{x}$.

To transform it to a SVP instance we use $d$ triples $(t_i, u_i, l_i)$ to construct a $d+2$ dimensional lattice $L(B')$ spanned by the rows of the matrix
$$B' =
\left(
  \begin{array}{cc}
     B\ \  &\  0\ \  \\
    \textbf{u}\ \ &\ q \ \  \\
  \end{array}
\right).
$$
Similarly, the vector $\textbf{x}' = (2^{l_1+1}(\alpha t_1 - u_1) \bmod q, ..., 2^{l_d+1}(\alpha t_d - u_d) \bmod q, \alpha, -q)$ belongs to the lattice $L(B')$.
 Its norm satisfies that $\|\textbf{x}'\| \leq q\sqrt{d+2}$,
  while the lattice determinant of $L(B')$ is $2^{d + \sum{l_i}}q^{d+1}$.
   This indicates that the vector $\textbf{x}'$ is a very short vector.
Note that this lattice also contains another vector $(-t_1, ..., -t_d, q, 0)\cdot B = (0, ..., 0, q, 0)$,
 which is most likely the shortest vector of the lattice.
Therefore we expect the second vector in a reduced basis of the lattice is equal to $\textbf{x}'$ with a ``good" chance for a suitably lattice reduction algorithm.
Then we acquire the secret key $\alpha$.

we can use LLL~\cite{Lenstra1982} or BKZ~\cite{Schnorr1994} algorithm to solve the SVP problem, while use Babai~\cite{Babai1986} algorithm or Enumeration technique to solve the CVP problem.


%转化为hnp问题,%引理
%构造格
%cvp/svp求解




%我还要去做的事情是:确定我能够获取到的bit数具体是多少bit
%我们攻击使用了多少个签名




















