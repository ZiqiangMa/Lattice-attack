% This is samplepaper.tex, a sample chapter demonstrating the
% LLNCS macro package for Springer Computer Science proceedings;
% Version 2.20 of 2017/10/04
%
\documentclass[runningheads]{llncs}
%

\usepackage{caption}
\usepackage{algorithm}
\usepackage{algorithmicx}
\usepackage{algpseudocode}
\usepackage{amsmath,amssymb}
\usepackage{graphicx}
%\usepackage{subfigure}
\usepackage{url}
\usepackage{multirow}
\usepackage{listings}
\usepackage{cite}
\usepackage{array}
\usepackage{enumerate}
%\usepackage[linesnumbered,ruled,vlined]{algorithm2e}
\usepackage{xcolor}
\usepackage{color}

% Used for displaying a sample figure. If possible, figure files should
% be included in EPS format.
%
% If you use the hyperref package, please uncomment the following line
% to display URLs in blue roman font according to Springer's eBook style:
% \renewcommand\UrlFont{\color{blue}\rmfamily}

%\newcommand{\authnote}[2]{{\bf \textcolor{blue}{#1}: \em \textcolor{red}{#2}}}
\newcommand{\authnote}[2]{}
\newcommand{\zf}[1]{\authnote{zf}{#1}}
%End of New packages and commands



\begin{document}
%
\title{Another Lattice Attack against the wNAF Implementation of ECDSA to Recover More Bits per Signature}
%
\titlerunning{Another Lattice Attack against ECDSA to Recover More Bits per Signature}
% If the paper title is too long for the running head, you can set
% an abbreviated paper title here
%
%\author{First Author\inst{1}\orcidID{0000-1111-2222-3333} \and
%Second Author\inst{2,3}\orcidID{1111-2222-3333-4444} \and
%Third Author\inst{3}\orcidID{2222--3333-4444-5555}}
%%
%\authorrunning{F. Author et al.}
%% First names are abbreviated in the running head.
%% If there are more than two authors, 'et al.' is used.
%%
%\institute{Princeton University, Princeton NJ 08544, USA \and
%Springer Heidelberg, Tiergartenstr. 17, 69121 Heidelberg, Germany
%\email{lncs@springer.com}\\
%\url{http://www.springer.com/gp/computer-science/lncs} \and
%ABC Institute, Rupert-Karls-University Heidelberg, Heidelberg, Germany\\
%\email{\{abc,lncs\}@uni-heidelberg.de}}
%%
\maketitle              % typeset the header of the contribution
%
\begin{abstract}
This paper presents a practical lattice attack on ECDSA which is implemented %in the OpenSSL,
 with the windowed Non-Adjacent-Form (wNAF) representations to compute the scalar multiplication over elliptic curves.
Compared with the existing lattice attacks in the literature,
    our method recovers the private key  from side channels
            by extracting more information per signature.
In particular, in the scalar multiplications,
 we monitor the invert function through cache side channels,
    in addition to the functions of double and add which are only exploited in the literature.
By monitoring the invert function,
    we obtain the signs of the wNAF representations of an ephemeral key.
% Through this improvement, we obtain the extra information about the sign of the wNAF representations the ephemeral key.
The ``double-add-invert'' chain obtained from the cache side channels,
    is disposed to extract the information of the ephemeral key at the position of the non-zero digits.
Theoretically, $153.2$ bits of the ephemeral key will be extracted per signature
    for 256-bit ECDSA on a prime field,
    much more than existing methods.
Then,
    we convert the problem of recovering the private key based on the extracted bits, to the Hidden Number Problem (HNP)
 which is solved by lattice reduction algorithms.
We implement the Flush+Flush cache-based side channel and applied it to ECDSA with the secp256k1 curve in OpenSSL 1.1.1b,
    to monitor the functions of double, add and invert.
In the experiments, the BKZ lattice reduction algorithm is used to solve the Shortest Vector Problem (SVP) of HNP instances.
The experimental results show that the private key is successfully recovered from $60$ signatures,
    if the Flush+Flush side channel is built without any error.
To the best of our knowledge, this is the first time to obtain and exploit the signs of the wNAF representations
    in the side channel attacks against ECDSA,
    and we demonstrate the potential to recover the private key from fewer signatures.
%This is better than some prior methods of only using partial information gained from the cache side channel.

\keywords{ECDSA \and windowed Non-Adjacent-Form \and OpenSSL \and Lattice Attack \and Hidden Number Problem \and Cache Side Channel}
\end{abstract}
%
%
%

\zf{is this attack on the latest openssl? if so, please list the version}

\section{Introduction}
\label{sec:intro}

The Elliptic Curve Digital Signature Algorithm (ECDSA) \cite{Johnson2001, ansi2005} is a digital signature algorithm over elliptic curves,
    widely used in many popular applications, such as TLS \cite{rfc5246} and OpenPGP \cite{openpgp2007}, smart card \cite{smartcard2007}, and Bitcoin \cite{bitcoin2008}.
The core operation of ECDSA is the scalar multiplication of a base point (or generator) over elliptic curves
    by a random nonce (or ephemeral key),
 and its semantical security relies on the computational intractability to find the ephemeral key
        for any given pair of a scalar multiplication and a base point, i.e., the elliptic curve discrete logarithm problem (ECDLP).

However, there are side channel attacks to obtain the information on the ephemeral key,
    during the scalar multiplication.
As long as some bits of the ephemeral key are leaked,
    the private key will be recovered \cite{Nguyen2001,HG2001,Nguyen2002,Nguyen2003}.
%
%The confidentiality of this nonce is very important
%As long as parts of the nonce are revealed through the side channels,
% some efficient attacks can be launched to recover the private key .
%
Howgrave-Graham and Smart \cite{HG2001} theoretically showed that DSA is vulnerable to such partial ephemeral key exposures in 2001.
Then, Nguyen and Shparlinski \cite{Nguyen2002} improved their method with further details and gave a provable polynomial-time attack.
More importantly, they extended this theoretical attack from DSA to ECDSA \cite{Nguyen2003}.
The basic idea is to reduce the problem of private key recovery to an instance of the Hidden Number Problem
(HNP), which is further reduced to the Closest Vector Problem (CVP) or the Shortest Vector Problem (SVP) in a lattice, with the knowledge of consecutive bits of the ephemeral keys.
Brumley et al. employed this lattice attack to recover the ECDSA private keys from leaked least significant bits (LSBs) \cite{Brumley2009} or leaked most significant bits (MSBs) \cite{Brumley2011}.

%In 2003, Nguyen \cite{Nguyen2003} presented a polynomial-time algorithm that recovers the ECDSA private key when a few consecutive bits of the random nonce $k$ are known for a number of signatures, which used the similar results for the DSA \cite{Nguyen2002}.


With the Flush+Reload cache side channels \cite{flushreload},
	Benger et al. \cite{Benger2014} obtained the position of the least non-zero digit to infer the LSBs of the ephemeral key.
This side channel exploits
	the windowed Non-Adjacent Form (wNAF) implementation \cite{GORDON1998129,Miyaji1997,Koyama1002,Solinas2000}
	for the scalar multiplication	
		of a known point $G$ by the the random ephemeral key $k$.
This exploitation works for the elliptic curves over a prime field $\mathbb{F}_{p}$
			in ECDSA, which is implemented in OpenSSL\cite{openssl} and other cryptographic engines.
Then,
	by constructing an HNP instance solved by the lattice reduction algorithm, they recovered the ECDSA private key from about 200 signatures.
In 2015, Van de Pol et al. improved this attack by a more effective way of exacting information from the cache side channel \cite{Van2015},
 which exploited the information about the positions of higher half non-zero digits of the wNAF representations of the ephemeral key.
In the same year, Cao et al.~\cite{Cao2015} presented two lattice-based differential fault attacks against ECDSA with wNAF implementations.
In 2016, Goudarzi et al.~\cite{Dahmun2016}  extended this attack to the blinded ephemeral keys,
  which are protected by the addition of a random multiple of the elliptic-curve group order or by a random Euclidean splitting.
Then,
Fan et al. proposed a new way of extracting and utilizing information obtained from the Flush+Reload side channels.
 The problem of recovering the
secret key is then transformed to the Extended Hidden Number Problem (EHNP) which is also solved by lattice reduction algorithm.
The number of signatures needed is reduced to $4$ to recover the private key \cite{Fan2016}.
In 2017, Wang et al.~\cite{Wang2017} presented another lattice attack on ECDSA using
 a small fraction of information from the ephemeral key.
They exploited the positions of two non-zero digits together with the length of the wNAF representation of the ephemeral key to construct an HNP instance. They need 85 signatures to recover the private key.
%��һ��������Ҫ�޸�


%Benger et al. [22] proposed a method to extract the LSBs of the ephemeral key with the knowledge
%of double-and-add chain of the execution of scalar multiplication in ECDSA implementation via the
%Flush+Reload attack, being able to recover the private key of the secp256k1 curve using about 200 signatures
%
%It was first in 2001 that Howgrave-Graham and Smart [23] proposed
%their attack against DSA heuristically under the assumption that some consecutive leaked ephemeral
%key bits be known by side-channel attacks. In 2002, Nguyen and Shparlinski [24] analyzed their method
%in further details and gave a provable polynomial-time attack against DSA when some consecutive bits
%(e.g., the LSBs) of the ephemeral keys were leaked. They also extended their results to ECDSA [25] and
%managed to recover a 160-bit private key with 100 signatures each leaking 3 consecutive least significant
%bits. The basic idea is to reduce the key recovery problem to an instance of the hidden number problem
%(HNP), which can be further reduced to the closest vector problem (CVP) in a suitable lattice, with the
%knowledge of consecutive leaked ephemeral key bits. The best result of this method so far is achieved
%by Liu and Nguyen [26] in 2013, using the algorithm BKZ 2.0 [27], which is one of the best lattice
%reduction algorithms up to now. A 160-bit signature with 2-bit leakage of LSBs can be recovered with
%100 signatures.

%֮ǰ�Ĺ���������Щ���⣺1.2.3
%���ǵĹ�������������ʲô���飬��ϸ˵��������ʲô���õ���ʲô�����
%��˵��������˵ʵ��
%��֮ǰ��ȣ����ǵ�Ч��������ʲô�ط���
%Ȼ��summary���ף�

%ֻ�Ƕ����ݷ��������˸Ľ�����û�л�ø��������

The above works extract the information in the wNAF representation of the ephemeral key.
They tried to make better use of the data achieved through the side channels and construct more effective lattice attacks
 to recover the private key.
They extract the least non-zero digit \cite{Benger2014}, half non-zero digits \cite{Van2015}, all non-zero digits \cite{Fan2016} or two non-zero digits with the total length~\cite{Wang2017}.
In summary,
all information exploited in these attacks are the non-zero digits and the positions,
 and no more extra information about the ephemeral key is extracted from the side channels.
Although the lattice attacks are different in these works,
     no effort and improvement have been made to achieve and exploit more information.

In our paper, we also focus on ECDSA with the wNAF representation of the ephemeral key.
%, especially the implementation in the OpenSSL library.
 We propose a new method to recover the ECDSA private key from the information leaked through side channels.
We try to not only make better use of the information achieve from side channels,
  but also to extract different information from the side channels.
First, we analyse the implementation code of ECDSA in OpenSSL
 and find that
in the implementation of the scalar multiplication the invert function is another exploitable vulnerability.
It inverts a number so that the subtraction is replaced by addition and the space storing the precomputed points is reduced by half.
In the calculation, if the sign of current non-zero digit of the ephemeral key is opposite to the previous one, the invert function is called, and the absolute value of this digit is used to index the precomputed points. Otherwise, the invert function is not called.
It helps us to determine the sign of the non-zero digits of the wNAF representations.
 % by exploiting the \emph{$EC\_POINT\_invert()$} function we can improve the cache side channel attacks.
So we obtain the sign of the non-zero digits in the wNAF representations of the ephemeral key by adding a monitor to the invert function.
 Thus, we achieve more information about the ephemeral key compared to previous attacks.
After that, we manage to recover the consecutive bits at the position of every non-zero digits of the ephemeral key based on the information obtained from the cache side channels.
Finally, we construct an HNP \cite{boneh1996} instance making the best of the consecutive bits of the ephemeral key.
This problem is solved by lattice reduction algorithms  through converting to the SVP.

This attack is applied to the secp256k1 curve in OpenSSL 1.1.1b in this paper.
We choose the Flush+Flush~\cite{gruss2016flush} attack to monitor the functions of double, add and invert.
By monitoring the  invert function
	in addition to the double and add functions in OpenSSL. %	we obtain the ``double-add-invert'' chain.
We successfully extract whether each digit of the ephemeral key is zero or not,
 and determine the signs of the non-zero digits from this chain.
Then we use the BKZ \cite{Schnorr1994} algorithm to solve the HNP instance,
    and effectively recover the ECDSA private key.
If the result of the Flush+Flush attack is perfect, we need about $60$ signatures to recover the private key with a probability of $3\%$.

Through the cache side channel, we get the information about all non-zero digits of the ephemeral key, the positions and the signs.
Compared to previous works, our method obtain more information about the ephemeral key, i.e. the signs of all non-zero digits.
We extract $153.2$ bits on average per signature for 256-bit ECDSA.
In theory, the number of signatures that required is $2$ if a suitable lattice is constructed.
%It  can be applied to all the curves without any restriction.
%The sign information of non-zero digits cannot be directly applied to previous works.
%Also, our method does not rely on some special properties of the elliptic curves.
To make better use of the sign information of the non-zero digits,
 we exploit the information obtained about the wNAF representation to extract sequences of consecutive bits of $k$ to construct the lattice attack,
because the steps in existing works do not work for this information.
We have to obtain multiple consecutive bits of the ephemeral key by converting the wNAF representation to the binary representation.
And then, we use these consecutive bits to construct the HNP instance.
To the best of our knowledge, this is the first time to obtain and exploit the
signs of the wNAF representations in the side channel attacks against ECDSA,
    and previous works cannot be straightforwardly applied to this information.
%������Ҫ������ʲô��������ϸ����������ʲô�£�ʵ������ʲô��Ȼ��õ���ʲô���
%��֮ǰ�Ĺ�����ȣ���������Щ���º͸Ľ���Ȼ�����ܽ᱾�ĵĹ���
Our contributions are summarized as follows:
\begin{itemize}
  \item
  We present a new lattice attack to recover the private key of ECDSA with wNAF representations.
  First, we
    improve the cache side channel attacks by monitoring
      the invert function.
      It is the first work to exploit the sign of the non-zero digits of the ephemeral key in side channel attacks.
   Second, for the data from the cache side channel, we do not use them to construct the HNP instance directly,
   as in previous works.
    We obtain multiple consecutive bits of the ephemeral key by converting the wNAF representation to the binary representation.
Finally, we use these consecutive bits to construct the HNP instance.
The private key is recovered by solving the HNP instance using lattice reduction algorithm.
These steps are different from existing works.
  \item
    We apply our method to the secp256k1 curve in the latest version of OpenSSL.
     Through the cache side channel, $153.2$ bits of information per signature is obtained on average.
    The experiments show that $60$ signatures are enough to recover the private key with a probability of $3\%$.
\end{itemize}

The rest of this paper is organized as follows.
Section \ref{sec:background} presents some preliminaries.
Section \ref{sec:attack} provides the details about our attack,
and Section \ref{sec:impl&exper} shows the implementation details and the experimental results.
Section \ref{sec:discussion} contains some extended discussions.
Section \ref{sec:relatedwork} introduces some related works.
Section \ref{sec:conclusion} draws the conclusion.

%ECDSA���㷺Ӧ��
%������Բ������ɢ��������
%�㷨���������Ա�����
%������ʵ����ȴ������׷���©��
%��˺ܶ���ͨ�����ŵ���������ȡ����ʵ���ϵ�й¶����Ϣ��������Կ�ָ���
%Cache���ŵ������ڽ������õ��㷺Ӧ��
%
%OpenSSL��ʵ����Ӧ����㷺��ʵ�֡�
%���У����������ϵ���Բ���ߣ�ʹ�õ���wNAF�㷨ʵ�ֵı����˷���
%XX�꣬����˶����ַ�ʽ�ĵĹ���
%XX������ʹ��Flush+Reload������ȡ�˶������ݣ������˸񹥻�
%������
%�����ǵ������У�����ʹ���ˡ��������ˡ�����
%�ﵽ�ˡ�����
%�ܵ���˵�����ĵĹ������£�
%1��
%2��
%
%���µ�ʣ�ಿ��������֯���ڶ��¡�������


\section{Preliminaries}
\label{sec:background}
In this section, we present the relevant background knowledge about the attack.
First we describe the ECDSA and its implementation using the wNAF representation in OpenSSL.
Then we explain the attack method of the cache side channels.
Also, the hidden number problem and its corresponding lattice attack are introduced
 for processing the data obtained from cache side channels.

\subsection{The Elliptic Curve Digital Signature Algorithm}
\label{intro_ecdsa}
The Elliptic Curve Digital Signature Algorithm (ECDSA) \cite{Johnson2001, ansi2005} is the adaption of one step of the Digital Signature Algorithm (DSA) \cite{DSS186} from the multiplicative group of a finite field to the group of points on an elliptic curve.

Let $E$ be an elliptic curve defined over a finite field $\mathbb{F}_{p}$ where $p$ is prime.
$G \in E$ is a fixed point of a large prime order $q$, that is $G$ is the generator of the group of points of order $q$.
These curve and point parameters are publicly known.
The private key of ECDSA is an integer $\alpha$ that satisfies $0 < \alpha < q$,
 and the public key is the point $Q = \alpha G$.
Given a hash function $h$, the ECDSA signature of a message $m$ is computed as follows:
\begin{enumerate}
  \item
    Select a random ephemeral key $0 < k < q$.
  \item
    Compute the point $(x, y) = kG$, and let $r = x$ mod $q$;
    if $r = 0$, then return to the first step.
  \item
    Compute $s = k^{-1} (h(m) + r\cdot\alpha)$ mod $q$;
    if $s = 0$, then return to the first step.
\end{enumerate}

The pair $(r, s)$ is the ECDSA signature of the message $m$.
For ECDSA signature, the ephemeral key $k$ must be kept random and secret.
The equation in the third step shows the private key can be computed if $k$ is leaked.
If some signatures use the same $k$, or the random number generator in use has the predictability, attackers can obtain the private key directly.
Even if a portion of $k$ is known, the private key can be retrieved by lattice attacks.
Therefore, the target of most attackers is the scalar multiplication $kG$ expecting to get some effective bits information about $k$.

%Given the knowledge of the ephemeral key k and all the
%known information of (s, r,m), the secret key can be easily
%recovered by
%�� = r?1(s �� k ? H(m)) mod q .


\subsection{The Scalar Multiplication using wNAF Representation}
\label{intro_wnaf}
%wnaf ��ʲô
%K �������� P �Ƕ������� K�ϵ���Բ���� E(K)�ϵĵ㣬k��һ��������,������Բ�����ϵ�ļӷ���ʽ����P��������� k��, k ��Ϊ������ kP��Ϊ��Բ���ߵ�ˣ�������˷�
$G$ is a point defined on the elliptic curve over the finite field,
 and the scalar $k$ is a big integer.
Scalar multiplication $kG$ on the elliptic curve means that the point $G$ is added to itself $k$ times.
There are several algorithms to implement the scalar multiplication.
In OpenSSL, scalar multiplication in the prime field is implemented using the windowed Non-adjacent Form (wNAF) representation \cite{GORDON1998129,Miyaji1997,Koyama1002,Solinas2000} of the scalar $k$.
In wNAF, a number $k$ is represented by a sequence of digits which value is either zero or an odd number satisfied $-2^{w} < k_{i} < 2^{w}$,
 where $w$ is the window size. In this representation, any continuous non-zero digit interval is at least $w$ zero digits.
The value of $k$ can be expressed as $k = \sum{2^{i}\cdot k_{i}}$.
Algorithm \ref{alg:wnaf} introduces the concrete method for converting a scalar into its wNAF representation.

\renewcommand{\algorithmicrequire}{\textbf{Input:}}
\renewcommand{\algorithmicensure}{\textbf{Output:}}

 \begin{algorithm}[t]
        \caption{Conversion to wNAF Representation}
        \label{alg:wnaf}
        \begin{algorithmic}[1]
            \Require Scalar $k$, window size $w$
            \Ensure $k$ in wNAF: $k_0$, $k_1$, $k_2$, ...

            \State $i \gets 0$
            \While{$k > 0$}
                \If {$k$ mod $2 = 1$}
                    \State $k_i \gets k$ mod $2^{w+1}$
                    \If {$k_{i} \geq 2^{w} $}
                        \State $k_{i} \gets k_{i} - 2^{w+1}$
                    \EndIf
                    \State $k \gets k - k_{i}$
                \Else
                    \State $k_{i} \gets 0 $
                \EndIf
                \State $k \gets k/2 $
                \State $i \gets i+1 $
            \EndWhile
        \end{algorithmic}
    \end{algorithm}

%������ת����naf
%ת������μ�������˷�
When computing the scalar multiplication $kG$,
first, a window size $w$ is chosen.
Then precomputation and storage of the points $\{\pm G, \pm 3G, ..., \pm(2^{w - 1})G\}$ are executed.
After converting $k$ to the wNAF form, the multiplication $kG$ is executed as the Algorithm \ref{alg:kg} describes.


\renewcommand{\algorithmicrequire}{\textbf{Input:}}
\renewcommand{\algorithmicensure}{\textbf{Output:}}

 \begin{algorithm}[t]
        \caption{Implementation of $kG$ Using wNAF}
        \label{alg:kg}
        \begin{algorithmic}[1]
            \Require Scalar $k$ in wNAF: $k_0$, $k_1$, ..., $k_{l-1}$ and precomputed points $\{\pm G, \pm 3G, ..., \pm(2^{w - 1})G\}$
            \Ensure $kG$

            \State $Q \gets G$
            \For{$i$ from $l-1$ to $0$}
                \State $Q \gets 2\cdot Q$
                \If {$k_i \neq 0$}
                    \State $Q \gets Q + k_{i}G$
                \EndIf
            \EndFor
        \end{algorithmic}
    \end{algorithm}

\zf{I think alg1 and alg2 can be merged into one figure}

From the algorithm we can find that the \textbf{if-then} block is vulnerable.
An attacker can use a spy process to monitor the conditional branches through side channels.
Then he can get an ``double-add'' chain, and through it whether the value of $k_i$ is zero can be inferred.
The attacker can use these information to retrieve the private key.

In the actual OpenSSL execution, the bit length of $k$ is set to a fixed value$\lfloor\log_{2}{q}\rfloor + 1$ of by adding itself $q$
or $2q$, which can resist the Brumley and Tuveri remote timing attack \cite{Brumley2011}.
In most cases, the multiplication is done as $(k + q)G$.
Also OpenSSL uses the modified wNAF representation instead of the generalized one as stated in Algorithm\ref{alg:wnaf} to avoid length expansion in some cases and to make exponentiation more efficient.
The representation of modified wNAF is very similar to the wNAF.
Each non-zero coefficient is followed by at least $w$ zero coefficients,
 except for the most significant digit which is allowed to violate this condition in some cases.
As the use of modified wNAF affects the attack results little,
 we only consider the case of the wNAF for simplification. %���һ����Ҫ�޸ģ�����һ����Ҫ�Ͷ�ƪ���¶Ա�����, ������Ҫ�ں��ʵ�λ�ü�������ʹ��256�����ߣ�window��СΪ3


%�������һЩ�������������ж� ����ͨ�����ŵ���ȡ��Ϣ��
%
%ʵ���е�ʹ��k+q
\subsection{The Scalar Multiplication in OpenSSL}
\label{intro_smulinssl}
In OpenSSL1.1.1b \cite{openssl}, the scalar multiplication with wNAF representation is implemented in the function \verb+ec_wNAF_mul()+.
The core computing part of the function is shown in Algorithm  \ref{alg:smopenssl}.

\renewcommand{\algorithmicrequire}{\textbf{Input:}}
\renewcommand{\algorithmicensure}{\textbf{Output:}}

\begin{algorithm}[t]
        \caption{The Implementation of The Scalar Multiplication in OpenSSL}
        \label{alg:smopenssl}
        \begin{algorithmic}[1]
            \Require Scalar $k$ in wNAF: $k_0$, $k_1$, ..., $k_{l-1}$ and precomputed points $\{G, 3G, ..., (2^{w} - 1)G\}$
            \Ensure $kG$
			
			\State $r \gets 0$, $is\_neg \gets 0$, $r\_is\_inverted \gets 0$
            \For{$i$ from $l-1$ to $0$}
            	\If {$r \neq 0$}
            		\State \Call{$EC\_POINT\_dbl$}{$r$}     {   }  $//$ $double$
                \EndIf
                \If {$k_i \neq 0$}
                	\State $is\_neg \gets (k_i < 0)$
                	\If {$is\_neg$}
                		\State $k_i \gets -k_i$
                	\EndIf
                	\If {$is\_neg \neq r\_is\_inverted$}
                		\If {$r \neq 0$}
                			\State \Call{$EC\_POINT\_invert$}{$r$}   {   }  $//$ $invert$
                		\EndIf                	
                    	\State $r\_is\_inverted \gets !r\_is\_inverted$
                	\EndIf
                	\If {$r = 0$}
                		\State $r \gets $ \Call{$EC\_POINT\_copy$}{$k_{i}G$}
                	\Else
                		\State $r \gets $ \Call{$EC\_POINT\_add$}{$r, k_{i}G$} {   }  $//$   $add$
                	\EndIf
                \EndIf
            \EndFor
            \State \Return r
        \end{algorithmic}
\end{algorithm}

In this function, it iterates the $k$ from the most significant digit to the least significant digit in its wNAF representation.
 In each digit, it performs a double operation (the \verb+EC_POINT_dbl()+ function).
  If the digit is not zero, it runs into the if-then block in line 6, and first determines whether an invert function is needed to execute.
   The invert function is to compute the inverse element of a number.
If the sign of the non-zero digit $k_i$ is opposite to the previous  non-zero digit,
 the invert operation (\verb+EC_POINT_invert()+) is performed (line 13),
which makes it only need the precomputation and storage of the points $\{G, 3G, ..., (2^{w - 1})G\}$ and saves a half of storage space.
Then it performs an addition operation (the \verb+EC_POINT_add()+ function) with  indexing from the precomputed points using the
absolute value of the digit.


From Algorithm \ref{alg:smopenssl}, it is obviously that the two conditional branches are vulnerable.
In each loop, the double operation is always performed.
But the add operation is only performed if the digit is not zero and the invert operation is only performed if the sign of the digit is opposite to previous one.
Therefore, if a spy process can obtain the execution sequence of
  the double and addition operations while this function is running,
one can determine whether each digit of $k$ is zero or not according to this sequence.
Also,  the execution sequence of the invert operation can be used to determine the sign of the non-zero digits combining the former sequence.



\subsection{Cache Side Channel Attacks}
\label{intro_cacheattack}
Cache side-channel attacks take advantage of the characteristic of cache activity
 that accessing data from caches is much faster than from memory.
Attackers exploit these time variations to deduce the operations of the target process and then infer the key information.
%Cache side-channel attacks are roughly divided into three categories: trace-driven, time-driven and access-driven attacks.
%The trace-driven attackers probe the variation of electromagnetic fields or power, to capture the profile of cache activities and deduce cache hits and misses \cite{ac2006}.
%In time-driven attacks \cite{Bonneau2006, Bernstein2005Cache}, the adversary passively measures the execution time to disclose the secret keys.
%In these two categories of attacks, the cache states are changed by the victim cryptographic engine (and the system activities),
% while an access-driven attacker \cite{Osvik2006, cachegame2011, flushreload} actively manipulates the cache states by running a malicious process that shares caches with the victim.
%
%In recent years, access-driven attacks have been favored by researchers,
% because it has higher attack accuracy which means it has the ability to accurately obtain the cache access of the target process.
%Also it has a wider application scenarios, such as the attacks in the virtualization environments and cloud environments.
Many attack methods are proposed \cite{Bonneau2006, Bernstein2005Cache, Osvik2006, cachegame2011, flushreload}
 since it is demonstrated feasible in theory \cite{Page2002}.
We introduce two typical access-driven attacks called Flush+Reload \cite{flushreload} and Flush+Flush \cite{gruss2016flush} that are relevant to our work.
%�ڽ�Щ����о��У����������Ĺ����ܵ��о����ǵ���������Ϊ����и��ߵĹ������ȣ��ܹ���׼ȷ�Ļ�ȡ��Ŀ����̶�cache�ķ���������Լ������ʹ�ó�������������⻯�ƻ������й����������������Ƕ����ֵ��͵ķ��������������н��ܣ������ֹ����ͱ�����ء�

The Flush+Reload attack \cite{flushreload} employs a spy process to monitor if the specific memory lines have been accessed or not by the victim process.
So this attack relies on sharing pages between the spy and the victim processes.
For attacks in the same machine, the spy can map the victim program file, the victim data file or shared libraries to its own address space to shared these pages with the victim.
While in the virtualizition environments, the page de-duplication technique of the VMM ensures the page sharing between the spy and the victim.
%���LLC����Ҫдһ��

One round of the attack consists of three phases:
\begin{itemize}
  \item
    \textbf{Flush:}
    In this phase, the attacker uses the \verb+clflush+ instruction to flush the desired memory lines out from the caches.
    This ensures that the next time these lines are accessed from the memory instead of the caches.
  \item
    \textbf{Waiting:}
    In this phase, the attacker waits a moment while the victim runs.
  \item
    \textbf{Reload:}
    This phase detects whether the victim accesses the memory lines flushed in the first phase during the waiting time.
    The attacker accesses the desired memory lines to reload them into caches, and measures the time it takes.
    Depending on the reloading time, the attacker can determine whether the memory lines are accessed by the victim.
    If the time is longer, it means that the attacker reloads the memory lines from the memory.
    That is the victim does not access these memory lines during the waiting time.
    Otherwise it means that these memory lines are already in the caches.
    So these memory lines are accessed by the victim during the waiting time.%��Ҫ����
\end{itemize}

The Flush+Flush attack \cite{gruss2016flush} also needs the shared pages.
Unlike measuring the reload time directly affected by the cache access,
 this attack relies on the execution time of the \verb+clflush+ instruction,
  which is affected by whether the data is cached or not.
The execution time of the \verb+clflush+ instruction is shorter if the data is not cached
 and higher if the data is cached.
So depending on this time, the attackers can determine the victim's cache activities.

The Flush+Flush attack also has three phases.
The first two phases are the same as in the Flush+Reload attack.
But in the third phase it flushes the cache again and measures the flush time instead of the reload time.
As the third phase flushes the caches, it doubles as the first phase for subsequent observations.

%��һ��F+F �� F+R���жԱȣ����Ӹ�Ч�����ȸ��ߣ�F�׶ο��Ե�������һѭ����F��
The execution time of the \verb+clflush+ instruction is less than the reload time on average.
    Also the first and third phases are merged together in Flush+Flush attack.
  Therefore the Flush+Flush attack is faster and can obtain more information than the Flush+Reload attack over the same time scale.
Thus, the Flush+Flush attack has a better accuracy.
Furthermore, compared with the Flush+Reload attack,
The Flush+Flush attack does not trigger prefetches when monitoring consecutive memory lines, otherwise it may affect the validity of the attack.


%��һ��Ҫ�Ͷ�ƪ���½��жԱȸĽ���f+r������Ҫ����
%f+r���㷨α���뿼���費��Ҫ���ӡ�


\subsection{The Hidden Number Problem and Lattice Attack}
\label{intro_hnp}
The Hidden Number Problem (HNP) is first presented by Boneh and Venkatesan \cite{boneh1996} in 1996.
It is used to recover the secret key of Diffie-Hellman algorithm, DSA and ECDSA, given some leaked consecutive bits of the ephemeral key.

Given a prime number $q$ and a positive $l$,
 and let $t_1, t_2, ..., t_d$ be randomly chosen, which are uniformly and independently in $\mathbb{F}_q$.
The HNP can be stated as follows:
  recover an unknown number $\alpha \in \mathbb{F}_q$ such that the known number pairs $(t_i, u_i)$ satisfy $v_i = |\alpha t_i - u_i|_{q} \leq q/2^{l+1}$ for $1 \leq i \leq d$,
   where $|\cdot|_q$ denotes the reduction modulo $q$ into range $[-q/2, ..., q/2)$.
If $|\alpha t - u|_{q} \leq q/2^{l+1}$ is satisfied,
the integer $u$ represents the $l$ most significant bits of $\alpha t$ which is defined as ${MSB}_l(\alpha t)$.


The HNP problem can be converted to the CVP/SVP in the lattice and solved by the lattice basis reduction algorithm.
Here we provide a briefly introduce to the lattice.
For more detailed references on lattice we refer to the literature \cite{latticereduction2000}.
Consider the Euclidean space $\mathbb{R}^{d}$
 and let $B = \{\mathbf{b_1}, \mathbf{b_2}, ..., \mathbf{b_z}\}$ be a set of linearly independent vectors in $\mathbb{R}^d$.
The set of vectors
$$
L = L(B) = \{\sum_{i=1}^{z}{\beta_i\mathbf{b_i}} \  |\ \beta_i \in \mathbb{Z} \}
$$
is the \textbf{lattice} generated by $B$.
The set $B$ is called a basis of $L$, and $L$ is spanned by $B$.
The number $z$ representing the number of vectors in $B$ is the dimension or rank of $L(B)$.
If $z = d$, the lattice $L(B)$ is a full-dimension lattice.

\noindent\textbf{Hard Lattice Problems} Since the lattice is a set of vectors,
 it has a shortest non-zero vector, and the norm this vector is known as the first minima and denoted by $\mathbf{\lambda_1}(L)$.
That is, $\mathbf{\lambda_1}(L) = \min\{\|\mathbf{u}\| \ |\  0 \neq \mathbf{u} \in L\}$, where $\|\mathbf{u}\|$ denotes the Euclidean norm of the vector $\mathbf{u}$.
The problem of finding a non-zero vector $\mathbf{v} \in L$ with minimal norm is called the shortest vector problem (SVP).
 While for a lattice $L$ and an arbitrary vector $\mathbf{v} \in \mathbb{R}^d$,
 the problem of finding a lattice vector $\mathbf{u} \in L$ of minimum distance from $\mathbf{v}$ is called the closest vector problem (CVP) similarly.
  In other words, finding a vector $\mathbf{u}$ satisfied $\|\mathbf{u}\| = \min\{\|\mathbf{u} - \mathbf{v}\| \ |\   \mathbf{u} \in L\}$.

Exploiting lattices to solve the HNP problem, we construct a $d+1$ dimensional lattice $L(B)$ spanned by the rows of the following matrix:
$$B =
\left(
  \begin{array}{ccccc}
    q & 0 & \cdots & 0 & 0 \\
    0 & q & \ddots & \vdots & \vdots \\
    \vdots & \ddots & \ddots & 0 & \vdots \\
    0 & \cdots & 0 & q & 0 \\
    t_1 & \cdots & \cdots & t_d & 1/2^{l+1} \\
  \end{array}
\right).
$$
Considering the vector $\mathbf{h} = (\alpha t_1 \bmod q, ..., \alpha t_d \bmod q, \alpha /2^{l+1})$,
 it can be obtained by multiplying the last row vector of $B$ by $\alpha$ and then subtracting appropriate multiples of the first $d$ row vectors.
Thus the vector $\mathbf{h}$ belongs to $L(B)$.
We call $\mathbf{h}$ the hidden vector because the last coordinate of $\mathbf{h}$ discloses the hidden number $\alpha$.
Let the vector $\mathbf{u} = (u_1, ..., u_d, 0)$,
 so the distance between $\mathbf{h}$ and $\mathbf{u}$ is $\|\mathbf{h} - \mathbf{u}\| \leq q\sqrt{d+1}/2^{l+1}$.
While the lattice determinant of $L(B)$ is $q^d/2^{l+1}$,
 thus the vector $\mathbf{h}$ is very close to the vector $\mathbf{u}$.
Solving the CVP problem with imput $B$ and $\mathbf{u}$ reveals the vector $\mathbf{h}$,
 hence the private key $\alpha$ is retrieved.

Because solving the CVP instance requires exponential time in the lattice rank,
 we can use the embedding technique \cite{Nguyen1999} to transform it to a SVP instance which just requires the polynomial time to solve.
We construct a $d+2$ dimensional lattice $L(B')$ spanned by the rows of the matrix
$$B' =
\left(
  \begin{array}{cc}
    B & 0 \\
    \mathbf{u} & q/2^{l+1} \\
  \end{array}
\right).
$$
Similarly, the vector $\mathbf{h}' = (\alpha t_1 - u_1 \bmod q, ..., \alpha t_d - u_d \bmod q, \alpha /2^{l+1}, -q/2^{l+1})$ belongs to the lattice $L(B')$.
 Its norm satisfies that $\|\mathbf{h}'\| \leq q\sqrt{d+2}/2^{l+1}$,
  while the lattice determinant of $L(B')$ is $q^{d+1}/2^{2l+2}$.
   This indicates that the vector $\mathbf{h}'$ is a very short vector.
Note that this lattice also contains other vector $(-t_1, ..., -t_d, q, 0)\cdot B = (0, ..., 0, q/2^{l+1}, 0)$,
 which is most likely the shortest vector of the lattice.
So we expect the second vector in a reduced basis of the lattice is equal to $\mathbf{h}'$ with a ``good" chance for a suitably strong lattice reduction algorithm.
Then we can acquire the hidden nubmer $\alpha$ by solving the SVP problem.

%we can use LLL \cite{Lenstra1982} or BKZ \cite{Schnorr1994} algorithm to solve the SVP problem, while use Babai \cite{Babai1986} algorithm or Enumeration technique to solve the CVP problem.

%��һ��˵����cache�������Ի����Ϣ��Ȼ����ת����hnp��������⡣
For the ECDSA algorithm implemented by OpenSSL with the wNAF,
  the attackers can obtain some consecutive bit fragments of the ephemeral key $k$ through the cache side channels.
Then, the problem of recovering the secret key can be transformed to the HNP problem.
 The attackers can recover the secret key by solving the HNP problem using lattice reduction algorithms.


%1 ʲô�Ǹ�

%2 ʲô�Ǹ���������

%3 HNP��������ø������



\section{Attacking ECDSA}
\label{sec:attack}
In this section, we propose a new method to retrieve the private key of ECDSA.
First, we analyse the invert function in the scalar multiplication with wNAF representation in the ECDSA algorithm.
 Then we use the invert function for improving the cache side channel attacks, from which the sign of the non-zero digits of $k$ is determined.
After that, we make use of the obtained information to recover the consecutive bits at the position of every non-zero digits of $k$.
Finally, we construct a HNP instance by the consecutive bits and transform it to the problem of solving the SVP/CVP in some lattice with the lattice reduction algorithms.

\subsection{Attacking wNAF Through The Cache Side Channels}
\label{data_get}
First recalling the implementation of scalar multiplication in OpenSSL, it uses the invert function to
  compute the inverse element of a number.
Applying this function, the space of precomputed points is saved by half, only need to store $\{G, 3G, ..., (2^{w} - 1)G\}$.
As showed in Algorithm~\ref{alg:smopenssl},
it iterates $k$ from the most significant digit to the least significant digit.
In each digit it performs a double function.
While if the digit is not zero, it first determines whether the invert function is executed.
 If the sign of the digit is opposite to the prior one, the invert function is called.
Then it indexes from the precomputed points using the absolute value of the digit and performs an addition operation.

Originally, the vulnerability comes from the double and add function.
The double function is called at every digit, and this reveals the digit sequence.
The add function is called just when the digit is not zero,
  which makes it possible to distinguish whether the digit is zero or not.
  That reveals the position of all the non-zero digits.
However,
 the invert function is also vulnerable, because it is called conditionally, either.
Only in the condition that the sign of the non-zero digit is opposite to the previous, the invert function will be performed.
Because we know the first non-zero digit is positive,
   the sign of all the non-zero digit can be deduced based on the execution of the invert function.


We use a spy process to monitor one memory line of the code of the double, add and invert functions while computing the scalar multiplication.
The time is divided into slots, and in each slot
  the spy determines whether the three functions are performed or not by monitoring the cache hits/misses.
 Then we can obtain a "double-add-invert" chain.
According to the "double" and "add" in this chain, we can determine whether each digit of $k$ is zero or not.
  That is what the original Flush + Reload attack does.
Also, based on the "invert" in this chain, we can infer the sign of each non-zero digit.


Specific to the OpenSSL implementation (especially the OpenSSL1.1.1b used in this paper),
  the double, add and invert function are implemented using  \verb+EC_POINT_dbl()+, \verb+EC_POINT_add()+ and \verb+EC_POINT_invert()+.
 The original cache side channel attack monitors the \verb+EC_POINT_dbl()+ and \verb+EC_POINT_add()+ function.
So as in our attack, we improve the original cache side channel attack by adding a new monitor to the \verb+EC_POINT_invert()+ function.
We use a spy process to monitor one memory line of the code of the three functions.
 In each time slot
  the spy determines whether the three functions are performed or not.
Then we get a new "double-add-invert" chain through the cache side channel attack.

%首先回顾标量乘法的实现,原始实现。
%然后invert函数在里面是如何使用的,起到了什么作用。
%漏洞在哪里,怎么样去利用。
%OpenSSL中他们的实现,
%我们在攻击中怎么做的
%然后得到了什么


When we use the "double-add-invert" chain to extract the digits of $k$,
the "double" represents the double function is called,
and the "add" represents the double and add functions are called both.
The "invert" represents the invert function is called.
Therefore,
 the "double" appears meaning that $k_i$ is zero
and the "add" appears meaning that $k_i$ is not zero.
 Then we use the "invert" to determine the sign of the $k_i$.
The sign of the $k_i$ is related to the previous non-zero digit.
First, the "invert" comes out together with "add".
 Then, if the "invert" appears, it represents that the sign of this digit $k_i$ is opposite to the previous non-zero digit.
 While if the "invert" does not appear when "add" comes out, it represents that
the sign of this digit $k_i$ is the same as the previous non-zero digit.
 For the wNAF representation of $k$, the most significant digit is always positive.
 Thus we can determine the sign of all non-zero digits.
In this way, we obtain all the positions of the non-zero digits and the sign information of them.


% In our work, we implement the Flush+Flush attack to obtain the "double-add-invert" chain because it has a better precision.
%%这里还需要再多说两句,为什么要精度更高,因为reload时间长,多监控一个函数,在一个slot中花费时间更多, 容易造成失误。 flush的时间短,slot可以更短。


%openssl的实现是XXXX,
%传统的攻击对double 和add进行监控
%我们增加对invert进行监控
%然后如何得到digit的正负信息
%在我们的实现中,通过对比f_f和f+r 选择f+f进行攻击


%问题是:标量乘法运算的起始位置很重要。

%符号:sign
%这一节用不用一个图来说明add-double链


\subsection{Consecutive bits Recovery}
\label{data_proc1}
In this section, we introduce how to recover the consecutive bits at the position of every non-zero bits for the ephemeral key $k$, exploiting the information obtained from the side channel.

First, we denote the wNAF representation of $k$ as $k = \sum{k_{i}2^{i}}$,
 and the binary representation as $k = \sum{b_{i}2^{i}}$.
From the Algorithm \ref{alg:wnaf} we know that the position of $k_i$ in the wNAF representation is the same as $b_i$ in the binary representation.
 When we know the information about whether $k_i$ is zero and the sign of the non-zero $k_i$, we can simply determine some bits of $k$.
 For example if we obtain the sign of the first non-zero $k_j$, we can infer that $b_j$ is one and $b_i$ is zero for $0\leq i<j$.
 But for arbitrary non-zero digits, it can not determine whether the bit is zero or one.

 Let $m$ and $m + l$ be the position of two consecutive non-zero digits of the wNAF representation, and $w$ be the window size.
 That is $k_m, k_{m+l} \neq 0$ and $k_{m+i} = 0$ for all $0 < i < l$.
 We analyse the transformation method between the binary and wNAF representation, getting the following result:
 \begin{align}
 &b_{m+l} = \left\{
 \begin{aligned}
 	&0,\,\;\ \   k_m < 0 \\
 	&1,\,\;\ \   k_m > 0
 \end{aligned}
 \right.   \\
 &b_{m+i} = \left\{
 \begin{aligned}
 	&0,\,\;\ \   k_m > 0 \\
 	&1,\,\;\ \   k_m < 0
 \end{aligned}
 \right.
 ,\ \ \ \ \  w \leq i \leq l-1
 \end{align}
%这里的公式有漏洞,需要加上最低位即m最小时的bm

In this way, at the position of every non-zero digit we can obtain $l - w + 1$ consecutive bits of $k$.
For the wNAF representation, every non-zero digit is followed by at least $w$ zero values.
 The average number of non-zero digits of $k$ is $\lfloor\log_{2}{q}\rfloor /(w+2)$.
 The average distance between consecutive non-zero digits is $w+2$, i.e. on average $l = w + 2$.
 This means we can obtain $3$ consecutive bits on average at the every non-zero digit.
Thus, on average we can obtain $3\lfloor\log_{2}{q}\rfloor /(w+2)$ bits of the ephemeral key $k$ in total.
For the secp256k1 curve implemented in OpenSSL, $\lfloor\log_{2}{q}\rfloor = 256$, $w = 3$.
 Also, as introduced previous, the scalar multiplication uses $k+q$ instead of $k$ in most cases,
  so the total number of bits per signature we obtain is $3(\lfloor\log_{2}{q}\rfloor +1)/(w+2) = 154.2$.
In theory, two signatures would be enough to recover the 256-bit private key as $2\times 154.2 = 308.4 > 256$.

%这里再加一小段,理论上[文献x]说,只要知道连续的2bit信息,就能够用来构造格攻击恢复密钥。
%因此,对侧信道得到的数据使用我们的方法恢复出来的连续比特位,全部都可以被用到后续的攻击中

%首先,k的二进制表示和k的wnaf表示
%他们的位是对应的。
%当我们知道了digit的正负消息后,简单的,我们就可以判断出某些二进制位是多少了
%然后我们来看,怎么去确定
%得到一个公式:
%
%然后我们就得到了连续的bit信息
%通过一个签名,我们能够获得的bit数为:


\subsection{Constructing the Lattice Attack}
\label{data_proc2}
In this section we transform problem of recovering the private key to the HNP instance, and further convert to the CVP/SVP instance of a lattice.

To construct a HNP instance using arbitrary consecutive bits, we need to use the following theorem\cite{Nguyen2002}:
\begin{theorem}
 \label{theorem1}
There exists a polynomial-time algorithm which, given $A$ and $B$ in $[1, q]$, finds $\lambda \in Z^{*}_{q}$ such that
$$
|\lambda |_q < B  \ \  \text{and} \ \  |\lambda A|_q \leq q/B .
$$
\end{theorem}
The value of $\lambda$ can be found exploiting the continued fractions.

Recall the ECDSA signature, $s = k^{-1} (h(m) + r\cdot\alpha)$ mod $q$.
We rewrite it as
\begin{equation}
\label{sig}
\alpha rs^{-1} = k - s^{-1}h(m)  \ \  \text{mod} \ \ q.
\end{equation}
Then assume that we are given the $l_i$ consecutive bits of $k$, starting at some known position $j$.
 So $k$ is represented as $k = 2^{j}a + 2^{l+j}b +c$ for $0 \leq a \leq 2^l -1$, $0\leq b \leq q/2^{l+j}$ and $0 \leq c < 2^j$.
 We apply the theorem with $A = 2^{j+l}$ and $B = q2^{-j-l/2}$, to obtain $\lambda$ such that
$$
|\lambda |_q < q2^{-j-l/2}  \ \  \text{and} \ \  |\lambda 2^{j+l}|_q \leq q/2^{j+l/2} .
$$
Multiplying by $\lambda$ and plugging the value of $k$, Equation \ref{sig} transformed to
$$
\alpha r\lambda s^{-1} = (2^{j}a - s^{-1}h(m))\lambda +(c\lambda + 2^{l+j}b\lambda)  \ \  \text{mod} \ \ q.
$$

 Let
 \begin{equation}
 \label{tu}
 \left\{
 \begin{aligned}
 	&t_i = \lfloor r\lambda s^{-1} \rfloor_q    \\
 	&u_i = \lfloor (2^{j}a - s^{-1}h(m))\lambda \rfloor_q
 \end{aligned}
 \right.
 \end{equation}
We then have that
 \begin{equation}
\label{lattice}
    |\alpha t_i - u_i|_q < q/2^{(l/2-1)}
\end{equation}
This way, we transform to the a HNP instance.

In practice, OpenSSL uses $k+q$ as the ephemeral key. So the Equation \ref{tu} remains the same, but the Inequation \ref{lattice} turns into
 \begin{equation}
\label{lattice2}
    |\alpha t_i - u_i|_q < q/2^{(l/2-\log_{2}{3})}
\end{equation}
%上面这一部分,主要是引用的他人工作,因此需要重新整合语言,并且看看如何引用。

Note that, the Equation \ref{lattice2} represents that the $l/2-\log_{2}{3} -1$ most significant bits of $\lfloor\alpha t_i\rfloor$ is $u_i$, based on the definition of the HNP.
So it should satisfy that $l/2-\log_{2}{3} -1 \geq 1$, i.e. $l > 7$.
That means the length of the consecutive bits used to  construct the HNP instance should be larger than $7$,
although we could use all the partitions of the consecutive bits of the ephemeral key in theory.

Next we turn the HNP instance into the lattice problem.
we construct a $d+1$ dimensional lattice $L(B)$ spanned by the rows of the following matrix:
$$B =
\left(
  \begin{array}{ccccc}
    2^{l_1+1}q & 0 & \cdots & 0 & 0 \\
    0 & 2^{l_2+1} & \ddots & \vdots & \vdots \\
    \vdots & \ddots & \ddots & 0 & \vdots \\
    0 & \cdots & 0 & 2^{l_d+1}q & 0 \\
    2^{l_1+1}t_1 & \cdots & \cdots & 2^{l_d+1}t_d & 1 \\
  \end{array}
\right).
$$
Considering the vector $h = (2^{l_1+1}\alpha t_1 \bmod q, ..., 2^{l_d+1}\alpha t_d \bmod q, \alpha)$,
 it can be obtained by multiplying the last row vector of $B$ by $\alpha$ and then subtracting appropriate multiples of the first $d$ row vectors.
Thus the vector $h$ belongs to $L(B)$.
Let the vector $u = (2^{l_1+1}u_1, ..., 2^{l_d+1}u_d, 0)$,
 so the distance between $h$ and $u$ is $\|h - u\| \leq q\sqrt{d+1}$.
While the lattice determinant of $L(B)$ is $2^{d + \sum{l_i}}q^d$,
 thus the vector $h$ is very close to the vector $u$.
Solving the CVP problem with imput $B$ and $u$ reveals the vector $h$,
 hence the private key $\alpha$ is retrieved.

To transform it to a SVP instance we construct a $d+2$ dimensional lattice $L(B')$ spanned by the rows of the matrix
$$
\left(
  \begin{array}{cc}
    B & 0 \\
    u & q \\
  \end{array}
\right).
$$
Similarly, the vector $h' = (2^{l_1+1}(\alpha t_1 - u_1) \bmod q, ..., 2^{l_d+1}(\alpha t_d - u_d) \bmod q, \alpha, -q)$ belongs to the lattice $L(B')$.
 Its norm satisfies that $\|h'\| \leq q\sqrt{d+2}$,
  while the lattice determinant of $L(B')$ is $2^{d + \sum{l_i}}q^{d+1}$.
   This indicates that the vector $h'$ is a very short vector.
Note that this lattice also contains other vector $(-t_1, ..., -t_d, q, 0)\cdot B = (0, ..., 0, q, 0)$,
 which is most likely the shortest vector of the lattice.
So we expect the second vector in a reduced basis of the lattice is equal to $h'$ with a "good" chance for a suitably strong lattice reduction algorithm.
Then we can acquire the secret key $\alpha$.

we can use LLL or BKZ algorithm to solve the SVP problem, while use Babai algorithm or Enumeration technique to solve the CVP problem.
%latticeCVP这一部分和第二章背景知识里重合很多,需要重新组织语言,同时和他人工作类似,需要区分开。

%转化为hnp问题,%引理
%构造格
%cvp/svp求解




%我还要去做的事情是:确定我能够获取到的bit数具体是多少bit
%我们攻击使用了多少个签名





















\section{Implementation and Experiments}
\label{sec:impl&exper}
In this section, we apply our method introduced in Section \ref{sec:attack} to attack the secp256k1 curve.
  We implement the Flush+Flush attack  to obtain
the cache side channel information.
the details of the implementation and the result of attacking the elliptic curves are provided.
Then we implement the lattice attack using the BKZ algorithm,
 and demonstrate the experiments with different parameters.
Finally, we compare our method with the previous attacks.

%这一章我们展示攻击的具体实现过程,
%对secp256k1进行攻击,曲线的window是3
%实现平台

\subsection{The Flush+Flush Attack}
\label{ffattack}
We test the Flush+Flush attack on an HP Elite 8300 running Ubuntu 16.04.
The machine features an Intel Core i7-3300 processor with four execution cores and a 8 MB LLC.
The attack target is the ECDSA implemented in OpenSSL1.1.0h, which uses wNAF representation in the scalar multiplication.
For the experiment, we use the curve secp256k1 for ECDSA.
Followings are the implementation details for the Flush+Flush attack.

%平台

\noindent\textbf{Get virtual address. }
In this attack, we use a spy process to monitor the  \verb+EC_POINT_dbl()+,  \verb+EC_POINT_add()+ and \verb+EC_POINT_invert()+ functions in OpenSSL.
So we need to know the virtual addresses of the three functions.
First we get the offset of the code required to monitor in the dynamic library.
Then we use the \verb+mmap()+ function to load the page where the monitored function code resides into the virtual address space of the spy process.
The \verb+mmap()+ function will return the initial address of this page so that we can get the virtual address of the monitored function based on the offset and the address of this page.
Alternative way is to use the \verb+dl_iterate_phdr()+ function to get the initial address of the dynamic library when loaded into the address space.

To determine offset of the memory lines in the  dynamic library  we built OpenSSL with debugging symbols.
 These symbols are not loaded at run time and do not affect the performance of the code.
The debugging symbols are, typically, not available for attackers,
 however their absence would not present a major obstacle to a determined attacker who could use reverse engineering [16].



%获取虚拟地址

%首先获取所需监控函数代码在动态库中的偏移量
%接着获取函数代码虚拟地址,有两种方法
%使用mmap()函数,将所监控函数代码所在页加载入地址空间中,获取到函数代码的虚拟地址
%或编译时链接动态库,运行时使用dl_iterate_phdr()函数,获取到动态库加载入地址空间的起始地址,根据偏移量,获得函数代码的虚拟地址
%
%
%
%Victim:通过调用OpenSSL的crypto.so库来进行ECDSA签名运算。
%Attacker:使用和victim相同的crypto库,对victim使用的点加和倍点函数以及EC_POINT_invert()代码所在的cache行进行监控。



\noindent\textbf{Threshold.}
We monitor the execution time of the \verb+clflush+ instruction to flush the monitored functions.
If the time is larger than the threshold, it means the memory line is accessed by the victim.
Otherwise, the memory line is not accessed.
The thresholds are calculated for every monitored functions.
For each address of the monitored functions, we record the time of flushing the cache 1000 times, and take the time larger than 99 percent samples plus 6 as the threshold of this address.
The thresholds are recalculated every time before the attack is triggered.


%阈值
%对每个需要监控的地址都计算各自的阈值
%每次监测都重新计算阈值
%对每个地址,监控1000次flush花费的时间,取99%以上样本所花的最大时间加上6作为该地址的阈值


\noindent\textbf{Trigger the attack.}
We monitor the execution time of the address of the three functions.
When any one of the time is larger than the corresponding threshold,
 we start to record execution time, which means the attack starts.
The record stops when the number of consecutive execution time of all the monitored addresses less than the threshold  is larger than 200.
If the number of record is less than 1000, we re-record the execution time.
Then we have a valid record.
 Due to the  disturbance of noise,
  the valid record does not necessarily contain the activities of the monitored functions in the scalar multiplication.
So we continuously obtain 10 valid records.
%触发
%1) 某一个监控地址的flush时间大于阈值以后,开始记录;
%2)所有地址连续小于阈值的数量大于200后,停止记录;
%3)如果记录数据量小于1000,重新开始记录;
%4)连续记录10次。


\noindent\textbf{Time slot.}
For the attack, the spy process divides time into time slots of approximately 1000 cycles.
In each slot, the spy flushes the memory lines in the add, double and invert functions (\verb+EC_POINT_dbl()+,  \verb+EC_POINT_add()+ and \verb+EC_POINT_invert()+) out of the caches.
Then time slot length is chosen to ensure that the three functions only execute once.
This allows the spy to correctly distinguish consecutive doubles.

\noindent\textbf{Initial Point Judgement.}
In order to precisely recover the sign of the ephemeral key, we need to determine which sample means the start of the scalar multiplication.
We can determine the
start of the scalar multiplication
by combining the double-and-add chain and the profiles of the double and add functions with wNAF representation.
Also we can determine the start position by monitor the code of copy function in OpenSSL.
 We note that when the computation of scalar multiplication starts, OpenSSL performs the copy function instead of the add function.
 Thus we can monitor the copy function, and when the copy function is called in the time slot and the add function is called in the next slot, it means that the scalar multiplication starts.

%起始判断
%对标量乘法代码分析,在对k的最高位处理时,并不是使用的点加,而是copy,因此,可以对copy函数进行监控,当一个时间槽内进行了倍点和copy操作,可以认定标量乘法的开始。
%如果不监控copy,也可以根据点加倍点的发生,去排除错误的触发;也可以根据NAF编码时点加倍点的特征来判断起始。

\subsubsection{Experiment Results}
Figure \ref{} shows a fragment of the output of the spy when OpenSSL performs ECDSA with the secp256k1 curve.

Then explain the figure.

%10次实验
We totally run 10 experiments, i.e. monitoring the ECDSA signature 10 times.
The accurate rate is defined as the value of the actual number of bits obtained divided by the ideal number of bits available.
Thus, from these experiments the average accurate rate of the data obtained from the Flush + Flush attack is xx\%.



%对这个示例进行解释:
%什么符号代表哪个操作
%sample大于阈值xx 表示在这个时间槽内,函数被执行了。
%然后,for example, 对于时间槽1, 什么被访问了,代表哪个函数被执行了,因此对应于0或者1,
%然后时间槽x代表1,
%接着说时间槽x2 代表1,同时invert调用了,说明该位符号和之前相反,
%时间槽x3,未调用invert,符号位和之前相同。
%
%时间槽x4可能是噪声,对噪声的处理
%
%和真实的k相比,正确率是x\%
%
%用ff攻击一条曲线,得到结果,得到double add invert 链,进行恢复
%然后恢复成功率是多少?
%
%
%结果


\subsection{Lattice attack}
\label{latticeattack}
%使用完美的侧信道结果
%我们的结果是:
%进行分析
%分析为什么我们没有利用全部的信息,理论上怎样,实际怎样
We apply our attack to the curve secp256k1 in our experiments, and we assume that the Flush + Flush attack is perfect, that is we can correctly obtain the double-add-and-invert chain and recover all the information about the digits of the ephemeral key it contains.

 To solve the SVP problem we use the BKZ algorithm implemented in \verb+fplll+ \cite{fplll}.
This implementation is only efficient for the block size less than 35.
So for the larger block size, we use the BKZ-2.0 \cite{bkz2} algorithm.
%In tuning BKZ2.0, we use the following strategy, at the end of every round we determined whether we had already solved for the private key, if not we continue, and then gave up after ten rounds.
To solve the CVP problem, we apply a pre-processing of either fplll or BKZ-2.0.
When applying pre-processing of BKZ-2.0 we limit to only one round of execution.
Then we apply an enumeration technique to search the closest vector.
We restrict the number of nodes in the enumeration tree to $2^{29}$,
   so as to ensure the enumeration did not go on for an excessive amount of time in the cases where the solution vector is hard to find (this mainly affected the experiments in dimension greater than 150).

We want to find the optimal strategy for our attack in terms of the following parameters:
\begin{itemize}
 \item  SVP or CVP
 \item  lattice dimension
 \item  block size of BKZ
 \item  the minimal value of $l$ (length of the consecutive bits of $k$)
 %\item  pruning strategy (optional)
 %\item  total signature number
\end{itemize}

Thus we perform a number of experiments with different values of the parameters,
and compute the probability of success in each case.
The minimal length of the consecutive bits of $k$ is ranged from $9$ to $11$.
Because in Section \ref{sec:attack} the HNP problem introduced requires $l/2 - \log_{2}{3} > 1$, the value of $l$ should be larger than $8$.
The block size in BKZ is ranged from $10$ to $40$ by step being $10$.
The number of consecutive bits for constructing the lattice denoted as $d$ is ranged from $50$ to $230$,
i.e. the dimension of the lattice for the SVP or CVP is $d + 1$ or $d + 2$.

Table \ref{svp9} shows the probability of success for different dimensions and block sizes of solving the SVP instance in the
             case that the length of the consecutive bits is larger than 9 and no pruning strategy is used.
As shown from the table, only need $60$ fragments of consecutive bits we successfully retrieve the private key of a 256-bit ECDSA.
  For fixed  minimal length of consecutive bits,
 increasing the dimension generally increases the probability of success.
In some sense, as the dimension increases
     more information is being added to the lattice and this makes the desired solution vector stand out more.
Also, the higher block sizes perform with a higher probability of success,
 as the stronger reduction allows them to isolate the solution vector better.

Table \ref{svp8} and Table \ref{svp10} show the probability of success in the case that the length of the consecutive bits is larger than 8 and 10, respectively.
The probability of success also increases as the the dimension or the block size increase.
Moreover, while the minimal length of consecutive bits increases, the probability of success becomes higher.
It is because that more information is being added to the lattice making it easier to search for the desired solution vector.

\begin{table}
  \centering
  \caption{the probability of success by solving SVP ($l > 9$)}
  \label{svp9}
    \begin{tabular}{|c|c|c|c|}
    \hline
    \multirow{2}{*}{Dimension}&
    \multicolumn{3}{c|}{Block size} \\%&\multicolumn{3}\\ %
    \cline{2-4}
    & 10 & 20 & 30\\
    %\hline
  \hline
  50 & 0 & 0 & 0 \\
  \hline
  60 & 0 & 0 & 1.0 \\
  \hline
  70 & 0.8 & 1.2  & 1.3 \\
  \hline
  80 & 4.3 & 8.6 &  9.7 \\
  \hline
  90 & 14.4 & 22.8 & 23.7 \\
  \hline
  100 & 21.0 & 33.6 &  \\
  \hline
  110 & 16.3 & 36.7 &  \\
  \hline
  120 & 21.7 & 35.0 &  \\
  \hline
  130 & 24.0 & 45.7 &  \\
  \hline
  140 & 29.0 & 46.3 &  \\
  \hline
  150 & 32.3 & 49.3 &  \\
  \hline
  160 & 27.7 & 49.3 &  \\
  \hline
  170 & 39.0 & 56.7 &  \\
  \hline
  180 & 37.3 & 57.0 &  \\
  \hline
  190 & 39.3 & 63.5 &  \\
  \hline
  200 & 46.0 & 66.0 &  \\
  \hline
  210 & 51.7 & 66.0 &  \\
  \hline
  220 & 51.3 & 72.5 &  \\
  \hline
  230 & 58.7 & 73.5 &  \\
  \hline
    \end{tabular}
\end{table}

%\begin{table}[b]
%  \centering
%\begin{tabular}{|c|c|c|c|}
%  \hline
%  % after \\: \hline or \cline{col1-col2} \cline{col3-col4} ...
%  dimension & 10 & 20 & 30 \\
%  50 & 0 & 0 & 0 \\
%  60 & 0 & 0 & 1.0 \\
%  70 & 0.8 & 1.2  & 1.3 \\
%  80 & 4.3 & 8.6 &  9.7 \\
%  90 & 14.4 & 22.8 &  \\
%  100 & 21.0 & 33.6 &  \\
%  110 & 16.3 & 36.7 &  \\
%  120 & 21.7 & 35.0 &  \\
%  130 & 24.0 & 45.7 &  \\
%  140 & 29.0 & 46.3 &  \\
%  150 & 32.3 & 49.3 &  \\
%  160 & 27.7 & 49.3 &  \\
%  170 & 39.0 & 56.7 &  \\
%  180 & 37.3 & 57.0 &  \\
%  190 & 39.3 & 63.5 &  \\
%  200 & 46.0 & 66.0 &  \\
%  210 & 51.7 & 66.0 &  \\
%  220 & 51.3 & 72.5 &  \\
%  230 & 58.7 & 73.5 &  \\
%  \hline
%\end{tabular}
%  \caption{the SVP result}\label{svp9}
%\end{table}

Table \ref{svp8, svp9, svp10} show the probability of success when solving the CVP instance.
As shown from the table,
  the number of fragments of consecutive bits that we need to retrieve the private key of a 256-bit ECDSA is only $60$, less than the result of the SVP instance with the low block size.
 The probability of success also increases as the the dimension or the block size increase.
Moreover, while the minimal length of consecutive bits increases, the probability of success becomes higher.
%再加一段或几句CVP结果和SVP结果进行比较。


%The results of the SVP and CVP experiments (Appendix A) show that for fixed  minimal length of consecutive bits,
% increasing the dimension generally increases the probability of success.
%In some sense, as the dimension increases more information is being added to the lattice and this makes the desired solution vector stand out more.
%The higher block sizes perform better,
% as the stronger reduction allows them to isolate the solution vector better.
%While the minimal length of consecutive bits increases, the the probability of success become higher.
%Also we see that extensive pre-processing of the basis with more complex lattice reduction techniques provides no real benefit.


\begin{table}[!t]
  \centering
   \caption{Comparison with previous attack methods}\label{compare1}
\begin{tabular}{|c|c|c|c|c|}
  \hline
  % after \\: \hline or \cline{col1-col2} \cline{col3-col4} ...
  methods & side channels & HNP or EHNP & number of bits & number of signatures \\
  \hline
  Benger et al. \cite{Benger2014} & Flush +Reload & HNP & 2 & 200 \\
  \hline
  van de Pol et al. \cite{Van2015} & Flush +Reload & HNP & 47.6 & 13 \\
  \hline
  Fan et al. \cite{Fan2016} & Flush +Reload & EHNP & 105.8 & 4 \\
  \hline
  Wang et al. \cite{Wang2017} & Flush +Reload & HNP & 2.99 & 85 \\
  \hline
  ours & Flush + Flush & HNP & 154.2  & 60 \\
%  \hline
%  190 & 39.3 & 63.5 & & \\
  \hline
\end{tabular}
\end{table}

\subsection{Comparison with Other Attack Methods}
\label{compare}

 Generally,  to attack the ECDSA implementation with the wNAF representation,
   the attackers target the implementation of scalar multiplication
      and use cache side channel attacks to achieve the information about the ephemeral key $k$.
Through side channels attackers can get a "double-and-add" chain for the scalar multiplication.
However,
 it is hard to directly recover the whole ephemeral key just depending on the "double-and-add" chain.
From the obtained information it can recover partial bits of the ephemeral key.
Hence attackers exploit the incomplete information of $k$ to transform the problem of private key recovery into one that can be solved by lattice, such as HNP or EHNP problem.
 Then the attackers retrieve the private key by solving the HNP or EHNP problem through being converted to the CVP/SVP problem in the lattice.

We compare the previous attack methods with ours in several aspects and the result is shown in Table \ref{compare1}.

Benger et al. \cite{Benger2014} got the least significant bits (LSBs) of the ephemeral key through the Flush + Reload attack
although this is not all the information obtained from the side channel.
Then they used the LSBs of many signatures to construct a HNP problem and
successfully retrieved the private key by solving the CVP/SVP instance of a specific lattice converted from the HNP problem.
The number of bits extracted from each signature is very small, only an average of 2 bits information can be obtained.
Thus made it require more than $200$ signatures to recover a 256-bit private key with the probability being $3.5\%$.
In our work, we exploit information from the cache side channel to extract the consecutive bits at the position of every non-zero bits for the ephemeral key.
All the consecutive bits can be used to construct the lattice attack.
Therefore, we can get much more bits per signature and the number of signatures is decreased for retrieve the private key.

van de Pol et al. \cite{Van2015} improved Benger's attack  relying on the property of some specific elliptic curves, that is the order $q$ of the base point is a pseudo-Mersenne prime which can be expressed as $2^n - \epsilon$, where $|\epsilon| < 2^p $, $p \approx n/2$.
With this property, this attack uses the information from the consecutive non-zero digits whose positions are between $p+1$ and $n$ extracted from the Flush + Reload attack, not only the LSBs.
It is able to extract $47.6$ bits per signature on average for the secp256k1 curve and recover the private key with 13 signatures.
Then they constructed a new HNP instance using this information and
successfully retrieved the private key.
Although this work greatly reduced the number of signatures needed to recover the private key,
the information from the cache side channel is not fully utilized.
But our work can use all the information from the cache side channel.
Moreover, their work is only effective to some special curves,
 but ours can be applied to all the curves without any restriction.
%This means that they can use about half of the information they got from the side channel.


Fan et al.\cite{Fan2016} extract all positions of digits from the Flush + Reload attack and take advantage of them to construct an EHNP instance.
They managed to obtain on average 105.8 bits per signature for the secp256k1 curve
 and only need 4 signatures to retrieve the private key with the probability being $8\%$.
 Compared with their work, ours obtains more information about the ephemeral key, i.e. the sign of all non-zero digits, from the side channel by analysing the OpenSSL implementation.
We can extract 154.2 bits information per signature.
Although it uses more signatures to retrieve the private key through our lattice attack,
theoretically the number of signatures needed is less than Fan's if a suitable lattice is constructed.

Wang et al. \cite{Wang2017} obtain the positions of two non-zero digits and the length of the wNAF representation of $k$ from the Flush + Reload attack.
They can obtain no less than 2.99 bits information per signature and exploit the HNP to recover the private key for the 256-bit curve using $85$ signatures.
Obviously, our method obtains more information and uses less signatures to recover the private key.

In summary, our method obtain the sign and the position of the ephemeral key $k$ from the Flush + Flush attack.
It can extract the largest amount of data, on average $154.2$ bits information per signature.
In theory, the number of signatures need to retrieve the private is $2$.
  However, due to the limit of our method of lattice construction, we need at least $60$ signatures, which is more than Fan's and van's.


%2017年,范团队还在science China information sciences 上发了一篇,使用HNP的文章,
%这是一个b类的期刊。(交叉综合新兴)
%我们比这个期刊的文章效果要好的。





























\section{Discussion}
\label{sec:discussion}

\subsection{Scalar Multiplication In Other Cryptographic Libraries}
Besides the OpenSSL library, there are many cryptographic libraries that implement the scalar multiplication.
But not all of them use the wNAF representation.
We introduce some implementations of the scalar multiplication in some popular cryptographic libraries.

The mbed TLS \cite{polarssl} is an open source SSL library licensed by ARM Limited.
It does not use the wNAF algorithm to represent the scalar.
It uses a comb method that generates a sequence of bit-strings to represent the scalar $k$ so that every bit-string represents an odd number based on modifying the method in \cite{Hedabou2004ACM}.
When computing the scalar multiplication, it traverses the sequence and directly indexes the value in the precomputed points to perform the addition.
It does not use the invert function, 
and this representation is hard to attack by the cache side channels because all the elements of the sequence are not zero. 

Libgcrypt is a cryptography library developed as a separated module of GnuPG.[3] It can also be used independently of GnuPG, but depends on its error-reporting library Libgpg-error[4].





\section{Related work}
\label{sec:relatedwork}
%��������ع����ֳ������飬һ���Dz�����Կй¶������һ����cache���ŵ�������
%����ʦ����������Ƿ���Ҫ������������Щ��ɾȥ��Щ�����߽ṹ������
%ǰ�潲�������ﻹ��Ҫ��˵��
\subsection{Partial Key Disclosure Attacks on DSA/ECDSA}
Several works have proposed different attacks that exploit the partial key disclosure to recover long-term private keys of the signature algorithms.
Some of them focus on how to exploit the partial information efficiently to construct some appropriate lattice attacks to recover the private key.
 And others pay more attention to how to efficiently obtain the information about the ephemeral key.

% first
Boneh and Venkatesan \cite{boneh1996} initially investigated to use the partial information of the ephemeral key to construct an HNP problem and recovered the private key of Diffie-Hellman by solving it using the lattice reduction algorithm.
Howgrave-Graham and Smart \cite{HG2001} extended the work of Boneh and Venkatesan \cite{boneh1996} and  showed how to recover the DSA private key by  constructing an HNP instance from leaked LSBs and MSBs of the ephemeral key.
Nguyen and Shparlinski \cite{Nguyen2002} improved their method  and gave a provable polynomial-time attack that
      knowing the $l \geq 3$ LSBs , the $l+1$ MSBs  or any $2l$ consecutive bits of a certain number of ephemeral keys was enough for recovering the DSA private key.
 They recovered  a $160$-bit DSA key only using the $3$ LSBs of a certain number of ephemeral keys.
 Further they extended these results to ECDSA~\cite{Nguyen2003}.
Liu and Nguyen \cite{Liu2013} improved the results that only 2 LSBs were required for breaking a 160-bit DSA key.
In 2014, Benger et al. \cite{Benger2014} extended the technique in \cite{Nguyen2002} to use a different length of leaked LSBs for each signature.
 They recovered the secret key of OpenSSL's ECDSA implementation for the curve \textbf{secp256k1} using about $200$ signatures.
Van de Pol et al. \cite{Van2015} exploited the property of the modulus in some elliptic curves so that they could use all of the information leaked in the top half of the ephemeral keys to construct the HNP instance, allowing them to recover the secret key after observing only $14$ signatures.
Fan et al. \cite{Fan2016} transformed the problem of recovering the secret key to the extended hidden number problem (EHNP)
  which was solved by the lattice reduction algorithm.
   Then the number of signatures needed was reduced to $4$.

%second
Brumley and Hakala \cite{Brumley2009} used an L1 data cache-timing attack to recover the LSBs of ECDSA ephemeral keys in OpenSSL 0.9.8k.
 They recovered a 160-bit ECDSA private key using the attack in \cite{HG2001} with collecting $2,600$ signatures ($8K$ with noise).
Analogously, Ac{\i}i{\c{c}}mez et al. \cite{Brumley2010} used an L1 instruction cache-timing attack to recover the LSBs of DSA ephemeral keys in OpenSSL 0.9.8l.
 It required $2,400$ signatures ($17K$ with noise) to recover a 160-bit DSA private key.
 Besides, both attacks required HyperThreading architectures.
In 2011, Brumley and Tuveri \cite{Brumley2011} mounted a remote timing attack on the implementation of ECDSA with binary curves in OpenSSL 0.9.8o.
 They obtained the MSBs of the ephemeral keys through the timing attack and recovered the private key after collecting information over 8,000 TLS handshakes.
In 2013, Mulder et al. \cite{Mulder2013} took advantage of a template attack to recover some LSBs of the ephemeral key.
They improved the Bleichenbacher's FFT-based (Fast Fourier Transform) attack by using BKZ for range reduction to solve the HNP problem.
Their attack could extract the entire private key using a 5-bit leak of the ephemeral key from 4 000 signatures.
Benger et al. \cite{Benger2014} and Van de Pol et al. \cite{Van2015} used the Flush+Reload technique to acquire the information of the ephemeral key.
Allan et al. \cite{Allan2016} improved the  results in \cite{Van2015} by using a performance-degradation attack to amplify the side-channel. The amplification allowed them to recover a 256-bit private key in OpenSSL 1.0.2a after observing only 6 signatures.
Genkin et al. \cite{Genkin2016} performed electromagnetic and power analysis attacks on mobile phones.
 They showed how to construct HNP triples when the signature uses the low $s$-value.
Pereida et al. \cite{Pereida2016} showed that the DSA implementation in OpenSSL is vulnerable to cache-timing attacks due to a programming error,
 and exploited the vulnerability to attack against SSH (via OpenSSH) and TLS (via stunnel).
In 2017 Zhang et al. \cite{Zhang2017} extended  the attack in ~\cite{Nguyen2003} to SM2 Digital Signature Algorithm (SM2-DSA), which is a Chinese version of ECDSA.


%Faug{\`e}re et al. \cite{jean2012}
%\cite{Gomez2019}
%\cite{}
%\cite{}
%\cite{}

\subsection{Cache Side Channel Attacks}
In 2002, Page \cite{Page2002} first described a theoretical cache based side channel attack on the collection of cache profiles for a large amount of chosen plaintexts.
Later Tsunoo et al. \cite{Tsunoo2003Cryptanalysis} proposed the first practical implementation of the cache attack on the DES cryptographic algorithm.
The first practical cache side-channel attack on AES was appeared in 2005 by Bernstein \cite{Bernstein2005Cache}
 and it statically analyzed the relation between the overall execution time and lookup table indexes affected by the cache behavior.
Bonneau and Mironov \cite{Bonneau2006} showed how to exploit cache collisions when indexing the AES lookup tables to recover the  AES secret key.
In 2006, Osvik et al. \cite{Osvik2006} presented two access-driven attacks: Evict+Prime and Prime+Probe.
Although the latter one proved to be significantly more efficient, both of them recovered the AES encryption key used by an OpenSSL server.
 Ac{\i}i{\c{c}}mez and Ko{\c{c}} \cite{ac2006} presented a trace-driven attack targeting AES that exploited internal table lookup collisions in the cache during the first round.
In 2011, Gullasch et al. \cite{cachegame2011} first presented a new attack which was later called the Flush+Reload is used to attack AES encryption by blocking the execution of AES after each memory access using the process scheduling algorithm.
In 2014, Irazoqui \cite{Irazoqui2014} exploit the Flush+Reload technique to attack the AES.

Percival~\cite{Percival2005CACHEMF} presented the first cache attack on RSA using the data cache collision.
Ac{\i}i{\c{c}}mez in \cite{Onur2007Yet} first exploited that the instruction cache leaked information when performing RSA encryption.
Brumley and Hakala \cite{Brumley2009} mounted an L1 data cache-timing attack to recover the LSBs of ECDSA ephemeral keys from the \verb+dgst+ command line tool in OpenSSL 0.9.8k and then used a lattice attack to recover a 160-bit ECDSA private key.
 While Ac{\i}i{\c{c}}mez et al. \cite{Brumley2010} used an L1 instruction cache-timing attack to recover the LSBs of DSA ephemeral keys from the same tool in OpenSSL 0.9.8l to recover a 160-bit DSA private key.
 In 2014 Yarom et al. proposed the Flush+Reload attack on RSA using the L3 cache \cite{flushreload}.
  Then, again Yarom et al. used the Flush+Reload technique to recover the secret key from a ECDSA signature algorithm \cite{yarom2014recovering}.

Cloud computing systems also can be attacked using the cache side channel attacks.
In 2009, Ristenpart et al. \cite{get-off-my-cloud} presented the methods to co-locate an attacker��s virtual machine (VM) with a potential victim��s VM with a success probability of 40\% in the Amazon EC2 cloud.
 Also the authors  showed that the cache side-channel attacks are both practical and applicable to real world scenarios by successfully recovering keystrokes from the co-resident victim��s VM.
This work provided a practical method to find the co-resident VMs, which was the basis of the cache attacks cross VMs because the caches were only shared with co-resident VMs.
For instance, Flush+Reload was used to attack AES \cite{Irazoqui2014} and RSA\cite{flushreload} in the cross-VM scenario.
Also the Prime+Probe method was also
adapted to work in virtualized environments by Zhang et al. \cite{YinqianZhang2012-cross-vm} and Liu \cite{liu2015last} to recover ElGamal encryption keys.
Irazoqui et al. \cite{fine2014} recovered AES keys in virtualized environments with Bernstein��s attack.
Benger et al. \cite{Benger2014} and Van de Pol et al. \cite{Van2015} demonstrated the viability of the Flush+Reload technique to recover ECDSA encryption keys.
Fan et al. \cite{Fan2016} proposed a new way of extracting and utilizing information from the Flush+Reload attack.
Flush+Flush \cite{gruss2016flush} and Prime+Abort \cite{disselkoen2017prime+abort} are the variants of Flush+Reload.
They exploited different methods to observe the cache access patterns to recover the private key and could be used in both local and virtualized environments.





%Several works have been presented to attack the ECDSA implementation with the wNAF representation.
% Generally, the attackers target the implementation of scalar multiplication
%  and use cache side channel attacks to achieve the information about the ephemeral key $k$.
%Through the cache side channels attackers can get a "double-and-add" chain for the scalar multiplication.
%However, because the wNAF representation of $k$ is a sequence of signed digits,
% it is hard to directly recover the whole ephemeral key just depending on the "double-and-add" chain.
%From the obtained information it can only determine the position of the non-digital bits and some least significants bits of the ephemeral key.
%Hence attackers exploit the incomplete information of $k$ to transform the problem of private key recovery into one that can be solved by lattice, such as HNP or EHNP problem.
% Then the attackers retrieve the private key by solving the HNP or EHNP problem through being converted to the CVP/SVP problem in the lattice.
%
%%����һ�������¼������裬�Ա����˷����й�����������ͨ��cache���ŵ�������ȡ�й�k��һЩ��Ϣ��
%%Ȼ��ͨ�����ŵ���ȡ����Ϣ���õ�k��ijЩbitλ����Ϣ������k����Ϣ��
%%ͨ����õ�����Ϣ�������˽Կ������ת��Ϊ����ͨ�������������⣬����ͨ������������������ȡ˽Կ��
%%ECDSA wNAF ������Կй¶����
%
%%�ҵ�һ����ͬ�ĵ㣬����������㣬���⼸������أ�Ȼ����з�����
%%
%%���ȶ���Щ���İ�ijЩ�ص���з���
%%��ÿ������ÿ�������ͱ��Ľ�����ϸ�Աȣ�
%%������ôʵ�ֵģ�����ʲô�������ﵽ��ʲôЧ��
%%���ĵ�������
%%�����ݻ�ȡ�������ݷ���������Ȼ�������бȽ�
%%��ͬ�㣬��ͬ��
%
%Benger et al. \cite{Benger2014} presented an attack to the ECDSA implementation with the wNAF representation in 2014.
%They got the least significant bits (LSBs) of the ephemeral key through the Flush + Reload attack
%although this is not all the information obtained from the side channel.
%Then they used the LSBs of many signatures to construct a HNP problem and
%successfully retrieved the private key by solving the CVP/SVP instance of a specific lattice converted from the HNP problem.
%The number of bits extracted from each signature is very small, only an average of 2 bits information can be obtained.
%Thus made it require more than 200 signatures to recover a 256-bit private key
%with the probability being 3.5\%.
%In our work, not only do we get more information from the cache side channel,
% but also we can exploit it to extract the consecutive bits at the position of every non-zero bits for the ephemeral key.
%All the consecutive bits can be used to construct the lattice attack.
%Therefore, we can get much more bits per signature and the number of signatures is decreased for retrieve the private key.
%
%In 2015, Van de Pol et al. improved Benger's attack.
%Their method relied on the property of some specific elliptic curves, that is the order $q$ of the base point is a pseudo-Mersenne prime.
%This prime can be expressed as $2^n - \epsilon$, where $|\epsilon| < 2^p $, $p \approx n/2$.
%With this property, this attack can use the information from the consecutive non-zero digits whose positions are between $p+1$ and $n$ extracted from the Flush + Reload attack, not only the LSBs.
%This means that they can use about half of the information they got from the side channel.
%Then they constructed a new HNP instance using this information and
%successfully retrieved the private key.
%According to this paper, it is able to extract 47.6 bits per signature on average for the secp256k1 curve and recover the private key with 13 signatures.
%Although this work greatly reduced the number of signatures needed to recover the private key,
%the information from the cache side channel is not fully utilized.
%But our work can use all the information from the cache side channel.
%Moreover, their work is only effective to some special curves,
% but ours can be applied to all the curves without any restriction.
%%the curves with the special property is just a small part of all the curves. This method of constructing the HNP instance can not be extended to other kinds of curves.
%
%Both of the two works exploited the HNP instance to retrieve the ECDSA private key.
%While Fan et al.\cite{Fan2016} proposed a new method to
% extract information from the Flush + Reload attack.
%It takes advantage of all positions of digits to construct an EHNP instance.
%The EHNP problem is solved in the same way as the HNP problem,
%being converted into the SVP problem on the lattice.
%They managed to obtain on average 105.8 bits per signature for the secp256k1 curve
% and only need 4 signatures to retrieve the private key with the probability being 8\%.
% Compared with their work, ours obtains more information about the ephemeral key, i.e. the sign of all non-zero digits, from the side channel by analysing the OpenSSL implementation.
%Furthermore, we transform the information of the wNAF representation to the binary information of $k$ to construct the lattice attack for the first time.
%We can extract xx bits information per signature.
%Although it uses more signatures to retrieve the private key through our lattice attack,
%theoretically the number of signatures needed is less than Fan's if a suitable lattice is constructed.
%
%2017�꣬���Ŷӻ���science China information sciences �Ϸ���һƪ��ʹ��HNP�����£�
%����һ��b����ڿ���(�����ۺ�����)
%��Ҫ������������жԱ��𣬻�������ƪ�͹��ˡ�
%���DZ�����ڿ�������Ч��Ҫ�õġ�
%
%
%%In 2017, \cite{Wang2017} give a lattice attack on the ECDSA implementation in the
%%latest version of OpenSSL which uses the wNAF method to implement the scalar multiplication, using
%%only a small fraction of information of the double-and-add chain of the ephemeral key.
%%we only need to know the positions of the second non-zero digit and the last non-zero
%%digit (from the higher index) together with the length of the chain rather than obtaining a perfect one
%%from the side-channel attack.
%%
%%
%%In 2015, \cite{Cao2015} presents two Lattice-Based Differential Fault Attacks Against ECDSA with wNAF Algorithm.
%%
%%In 2016, \cite{Dahmun2016}  extend the famous Howgrave-Graham and Smart lattice attack when the nonces are blinded by the addition of a random multiple of the elliptic-curve group order or by a random Euclidean splitting.
%
%%In 2015, Van de Pol et al. improved Benger's attack.
%%They also used the Flush + Reload method to attack the scalar multiplication.
%%They used the position information of the non-zero digits extracted from the side channel, not only the LSBs.
%%Their method relied on the property of some specific elliptic curves, that is the order $q$ of the base point is a pseudo-Mersenne prime.
%%This prime can be expressed as $2^n - \epsilon$, where $|\epsilon| < 2^p $, $p \approx n/2$.
%%With this property, this attack can use the information from the consecutive non-zero digits whose positions are between $p+1$ and $n$ extracted from the side channel.
%%This means that they can use about half of the information they got from the side channel.
%%Then they constructed a new HNP instant using this information and
%%successfully retrieved the private key.
%%According to this paper, it is able to extract 47.6 bits per signature on average for the secp256k1 curve and recover the private key with 13 signatures.
%%This work greatly reduced the number of signatures needed to recover the private.
%%Although it used half of the information more than the Benger's, the information from the cache side channel is not fully utilized.
%%Moreover, the curves with the special property is just a small part of all the curves. This method of constructing the HNP instance can not be extended to other kinds of curves.









\section{Conclusion and Future Work}
\label{sec:conclusion}
In this paper, we demonstrate a practical attack on the ECDSA algorithm implemented in OpenSSL with the scalar multiplication using the wNAF representation.
We improve the original cache side channel attack
 by adding an extra monitor to the invert function and get the extra information about the signs of all non-zero digits of the ephemeral key.
Then we exploit a new method to get the consecutive bits of the ephemeral key at each position of the non-zero digits through the ``double-add-invert'' chain obtained by the cache side channel attack.
From the side channel, we obtain $153.2$ bits of information per signature for the 256-bit ECDSA secp256k1 curve.
We construct a lattice attack using the HNP problem to recover the ECDSA private key.
This is the first work to obtain the information about the signs of the non-zero digits of the ephemeral key and make use of it to recover the ECDSA private key.
We implement our method to attack the secp256k1 curve.
The experiments show that we successfully recover the private key from only $60$ signatures.

In the future,
we will optimize this work to use fewer signatures and achieve higher success probability by finding ways to decrease the length of consecutive bits required and using more efficient lattice reduction algorithms such as the BKZ2.0.
Next, we will try to find a more efficient lattice construction to recover the private key,
    making full use of the information achieved (e.g., an EHNP-based solution).
Also, we will extend this attack to other cryptographic engines,
    and also the blinded ephemeral keys of ECDSA.





%\section{Acknowledgment}

%
% ---- Bibliography ----
%
% BibTeX users should specify bibliography style 'splncs04'.
% References will then be sorted and formatted in the correct style.
%
% \bibliographystyle{splncs04}
% \bibliography{mybibliography}
%

\bibliographystyle{splncs04}
%\bibliographystyle{alpha}
\bibliography{bib/Reference}


\end{document}
